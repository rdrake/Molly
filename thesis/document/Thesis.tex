% !TEX root = Thesis.tex
\documentclass[grad,openright]{uoit-thesis}

% !TEX root = Presentation.tex
\usepackage{fontspec}
\usepackage{graphicx}

% Nicer math
\usepackage{amsmath}
\usepackage{amsthm}
\usepackage{amssymb}
\usepackage{mathtools}
\usepackage{unicode-math}
\usepackage{subdepth}

% Pretty tables
\usepackage{booktabs}

% Complain when we do bad things
\usepackage{nag}

\usepackage{hyperref}

% Make the text worth looking at
\defaultfontfeatures{Ligatures=TeX}
\setmainfont[
	Kerning=Uppercase
]{Garamond Premier Pro}
%\setsansfont{Garamond Premier Pro}
\setmathfont{XITS Math}
%\setmathfont[range=\mathbfsfit/{greek,Greek,latin,Latin}]{Source Sans Pro}
\setmonofont[Scale=0.9]{Consolas}
\setsansfont{Source Sans Pro}

% amsldoc, Section 4.14.2
\providecommand{\abs}[1]{\lvert#1\rvert}
\providecommand{\norm}[1]{\lVert#1\rVert}

% Simplify some of our function declarations
\DeclareMathOperator{\similarity}{similarity}
\DeclareMathOperator{\df}{df}
\DeclareMathOperator{\tfidf}{tf-idf}
\DeclareMathOperator{\tf}{tf}
\DeclareMathOperator{\idf}{idf}
\DeclareMathOperator{\score}{score}
\DeclareMathOperator{\freq}{freq}
\DeclareMathOperator{\query}{query}

\definecolor{dark-gray}{gray}{0.50}

%\newcommand{\jeremycomm}[1]{\todo[color=green!40]{#1}}
%\newcommand{\markcomm}[1]{\todo[color=red!40]{#1}}
%\newcommand{\kencomm}[1]{\todo[color=blue!40]{#1}}
%\newcommand{\masoudcomm}[1]{\todo[color=yellow!40]{#1}}
%\newcommand{\toreviewers}[1]{\todo[color=purple!40,inline]{#1}}

\setbeamertemplate{navigation symbols}{}

\institution{\glsdesc{uoit}}
\degree{Master of Science (M.Sc.)}
\faculty{Science}
\department{Computer Science}
\gradmonth{June}
\gradyear{2014}
\supervisor{Dr. Ken Q. Pu}
\author{Richard J.I. Drake}
\title{Towards a Concurrent Implementation of Keyword Search Over Relational Databases}

\begin{document}
	% Some acronyms...
	\newacronym{rdbms}{RDBMS}{relational database management system}
	% Improve the empty set symbol.
\let\oldemptyset\emptyset
\let\emptyset\varnothing

\newglossaryentry{ndocs}
{
	type = symbols,
	name = {\ensuremath{N}},
	description = {number of documents in collection},
	sort = dm_ndocs
}

% Expect the function name and any arguments separated by commas
\newcommand{\prop}[2]{\ensuremath\textsc{#1}[#2]}
% Expect the relation name and command-separated list of attributes
\newcommand{\rel}[2]{\ensuremath\text{#1}(\text{#2})}

\newcommand{\relations}[1]{\prop{Rel}{#1}}
\newcommand{\attributes}[1]{\prop{Attr}{#1}}
\newcommand{\fields}[1]{\prop{Field}{#1}}
\newcommand{\name}[1]{\prop{Name}{#1}}
\newcommand{\keys}[1]{\prop{Key}{#1}}
\newcommand{\fks}[1]{\prop{FK}{#1}}
\newcommand{\bag}[1]{\prop{Bag}{#1}}
\newcommand{\docs}[1]{\prop{Doc}{#1}}
\newcommand{\uid}[1]{\prop{UID}{#1}}

\newcommand{\db}{\glssymbol{db}}
\newglossaryentry{db}
{
	type = symbols,
	name = {database},
	symbol = {\ensuremath D}, % I desire \mathbb{D}, not D.
	description = {set of \glspl{relation}},
	sort = rm_10_db
}

\newcommand{\relation}{\glssymbol{relation}}
\newglossaryentry{relation}
{
	type = symbols,
	name = {relation},
	symbol = {\ensuremath r},
	description = {set of named tuples},
	sort = rm_20_relation
}

\newcommand{\tuple}{\glssymbol{tuple}}
\newglossaryentry{tuple}
{
	type = symbols,
	name = {named tuple},
	symbol = {\ensuremath t},
	description = {ordered set of values},
	sort = rm_30_tuple
}

\newcommand{\attribute}{\glssymbol{attribute}}
\newglossaryentry{attribute}
{
	type = symbols,
	name = {attribute},
	symbol = {\ensuremath \alpha},
	description = {named column},
	sort = rm_40_attribute
}

\newcommand{\key}{\glssymbol{key}}
\newglossaryentry{key}
{
	type = symbols,
	name = {key},
	symbol = {\ensuremath \Kappa},
	description = {uniquely identifies a \gls{tuple} in a \gls{relation}},
	sort = rm_50_key
}

\newcommand{\sgraph}{\glssymbol{sgraph}}
\newglossaryentry{sgraph}
{
	type = symbols,
	name = {schema graph},
	symbol = {\ensuremath G},
	description = {graph representation of schema},
	sort = dm_00_sgraph
}

\newcommand{\egraph}{\glssymbol{egraph}}
\newglossaryentry{egraph}
{
	type = symbols,
	name = {entity group},
	symbol = {\ensuremath T},
	description = {forest of \glspl{tuple}},
	sort = dm_05_egraph
}

\newcommand{\dc}{\glssymbol{dc}}
\newglossaryentry{dc}
{
	type = symbols,
	name = {document collection},
	symbol = {\ensuremath C},
	description = {set of \glspl{doc}},
	sort = dm_10_dc
}

\newcommand{\terms}{\glssymbol{terms}}
\newglossaryentry{terms}
{
	type = symbols,
	name = {terms},
	symbol = {\ensuremath \Tau},
	description = {set of unique \glspl{term} in a \gls{dc}},
	sort = dm_20_terms
}

\newcommand{\doc}{\glssymbol{doc}}
\newglossaryentry{doc}
{
	type = symbols,
	name = {document},
	symbol = {\ensuremath d},
	description = {set of fields},
	sort = dm_30_doc
}

\newcommand{\q}{\glssymbol{q}}
\newglossaryentry{q}
{
	type = symbols,
	name = {search query},
	symbol = {\ensuremath q},
	description = {special case of \gls{doc}},
	sort = dm_35_q
}

\newcommand{\field}{\glssymbol{field}}
\newglossaryentry{field}
{
	type = symbols,
	name = {field},
	symbol = {\ensuremath f},
	description = {named sub-document in \gls{doc}},
	sort = dm_40_field
}

\newcommand{\term}{\glssymbol{term}}
\newglossaryentry{term}
{
	type = symbols,
	name = {term},
	symbol = {\ensuremath \tau},
	description = {unique term in \gls{dc}},
	sort = dm_50_term
}

\newcommand{\w}{\ensuremath w}
	
	% Automatically switches to Roman numerals
	\begin{preliminary}
		\maketitle

		% Title page is i, certificate of approval is ii, TOC is page...
		\setcounter{page}{3}
		
		\addcontentsline{toc}{chapter}{Abstract}
		\chapter*{Abstract}
			Vast amounts of data are stored in relational databases.  Traditionally, querying this data required a deep understanding of the underlying schema in addition to knowledge of a query language such as \gls{sql}.  We present a framework for the automatic, lossless transformation of data from the relational model to the document model.    By performing this transformation, users may locate information by using simple keyword queries.  We further this by implementing graph search, allowing users to automatically discover related facts of information.  The effects of performing graph search concurrently are explored.
			\\ \\
			\textbf{Keywords}:  relational database, full-text search, graph search, algorithms

		\tableofcontents
		
		\addcontentsline{toc}{chapter}{List of Tables}
		\listoftables
		
		\addcontentsline{toc}{chapter}{List of Figures}
		\listoffigures
		
		\addcontentsline{toc}{chapter}{List of Algorithms}
		\listofalgorithms
		
		\printglossaries
	\end{preliminary}
	
	%% Hypothesis, motivation
% Needs a clear outline of contributions

% Motivation -> Hypothesis -> Contributions -> 
% Why concurrent?  Why is keyword search important?

% Improve performance and allow more flexibility

% Why dataset chosen

% 5.5 -> threats to validity (external mostly)
% 6.2 -> limitations & future work (not just about tool, also about research questions left unanswered)
\chapter{Introduction}
	In this thesis, we are motivated by the following problems:
	
	\begin{itemize}
		\item Define a formal framework to describe data sets with relational structures and text components.
		\item Design a collection of expressive query operators for analyzing text relational data sets.
		\item Investigate implementation techniques to make the query operators performant for modern multi-core computers.
	\end{itemize}
	
	% !TEX root = Thesis.tex
\chapter{A Tale of Two Data Models}
\label{chap:tale-of-two-data-models}
	The term ``data model'' refers to a notation for describing data and/or information.  It consists of the data structure, operations that may be performed on the data, as well as a set of constraints placed on the data \cite{dbsys-06}.
	
	In this chapter we provide background and motivation for this thesis.  We will discuss the evolution of data models and their corresponding query languages.  We feel that modern day data sets call for a new data model with a new query language.
	
	We provide a formal definition of the relational data model, discuss its merits, its shortcomings, and contrast it to the document data model.  Contrary to the relational model, the document model permits fast and flexible keyword search without requiring explicit domain knowledge of the data.
	
	\chapter{Background \& Introduction}
	In this chapter, we provide a background to the motivation of this thesis.  We will discuss the evolution of data sets, their logical models, and the corresponding query languages.  We feel that  modern day data sets call for a new data model with a new query language.  This thesis is an attempt to make an incremental advance toward a new data model and new way of querying data.

	\section{Structured Data and Structured Language:  1970 -- 2000}
		The proliferation of database system research and development started with the disruptive invention of the relational data model by Edgar F.~Codd \cite{codd-79}.
		
		The invention of the relational data model was a significant achievement as it decoupled the task of data analytics from any one language, precisely and accurately described data sets across a variety of storage and analytical systems, and lead to the creation of the structured data analytics language known as \gls{sql}.
		
		\gls{sql} itself deserves further discussion; the relational data model provided a foundation upon which languages for data manipulation were designed.  One can describe their data set and operations using first-order logic and relational algebra, then realize it using \gls{sql}.
		
		The success of the relational model can only be appreciated when one looks at the continuous success of \gls{rdbms} such as Oracle and IBM DB2 which span over 3 decades without any significant decline.
		
		Since the 90s, the emergence of Business Intelligence \cite{bikm-02} furthered the development of \gls{rdbms} by specializing the relational data model to multidimensional data model \cite{colliat-96}.  The family of databases known as \gls{olap} produced a new query language known as \gls{mdx}.
		
		Both \gls{sql} and \gls{mdx} are highly structured query languages: they are completely described by their respective grammar and operational semantics.  Users who wish to harness the power of \gls{sql} and \gls{mdx} must be well versed in the languages themselves, and understand precisely the semantics of each syntactic constructs.
		
	\section{Text data and keyword search:  1970 -- 2014}
		Parallel to the development of the relational data base technology, research in the information retrieval has been focusing on text data \cite{salton-88, jones-72}.  Unlike the relational data, text data does not have much complex structure to its schema.  Thus, it's not immediately possible to design a rich set of data operators (as was the case for relational algebra).  Consequently, for text data, there is no structured query language like \gls{sql}.
		
		The research, thus, has been focusing on pattern matching queries using keyword search.  Though the semantics of keyword search is very simple, the statistical methods developed by the information retrieval community \cite{salton-88, robertson-09, dumais-88} have been extremely effective.  In fact, one can argue that the modern World Wide Web and its related commercial successes founded on the ideas of text databases and keyword search over Internet scale data sets.
		
	\section{Semi-structured Data and Query Languages:  1990 -- 2010}
		The growth of the World Wide Web popularized the usage of markup languages (such as \gls{html} and \gls{xml}) as the underlying Web content description.  Thus, researchers have designed data models \cite{suciu-98} to formalize the semantics of \gls{xml} and related data formats.  Subsequently, the logical characterization of \gls{xml} led the to design and implementation of XQuery \cite{xquery-10}, a navigational based query language for analysis of \gls{xml} data sets.
		
		Interestingly enough, \gls{xml} has proven to be inefficient as a data description format.  Nonetheless, a semi-structured data description language is highly sought after for message passing in Internet scale software systems.  Modern Web services are built on the concept of RESTful \gls{api} \cite{restful-11}, with semi-structured data message passing.  In order to minimize network overhead, \gls{xml} based message passing has been replaced by the more efficient data encoding standard of \gls{json} \cite{json}.
		
		The query language community responded to the popularization of \gls{json} encoded data sets with several query languages \cite{simeon-13} (for example Jaql \cite{ibm-jaql}) for \gls{json} data sets.
	
	\section{Hybrid data models and query languages: 2010 -- Present}
		With the explosive growth of social networks, we are witnessing the emergence of a new type of data sets.  These data sets exhibit the following properties:
		
		\begin{itemize}
			\item The data has relational characteristics: such as relationships of friends on Facebook, their preferences over different Web sites, and their account information.
			\item The data also has many text attributes: such as blog articles, or tweets on Twitter.
			\item The volume of data is often Internet scale.
		\end{itemize}
		
		The mixture of relational structure and rich text components of such data sets make them challenging for the purpose of data management and data analytics.  There has been several attempts to integrate keyword search from information retrieval with \gls{sql} \cite{banks-02, fuzzy-11, ir-03}.  However, these methods, thus far, suffer from scalability and restricted search capabilities.
		
		In this thesis, we are motivated by the following problems:
		
		\begin{itemize}
			\item Define a formal framework to describe data sets with relational structures and text components.
			\item Design a collection of expressive query operators for analyzing text relational data sets.
			\item Investigate implementation techniques to make the query operators performant for modern multi-core computers.
		\end{itemize}
		
	\section{Outline of Thesis}
		\cref{chap:tale-of-two-data-models} provides a formal definition of the relational data model and the document data model.  Rather than introducing a new hybrid model, our thesis is to provide efficient mappings between the two models, and thus allow the data to freely be exchanged between query operators of the domains.  We demonstrate that complex relational data can be encoded into a linked document space (and back).\todo{last sentence doesn't flow}
		
		\cref{chap:along-came-clojure} describes the details of our implementation the data transformation and query operators for graph search in the linked document space.  Our choice of utilizing a modern functional programming language for our implementation makes high degree of concurrency possible.
		
		\cref{chap:experimental-evaluation} provides further justification to our choice of data model mapping and the choice of programming language used.  Through a series of experiments, we see that our proposal allows a tight integration of the relational database engines and keyword search libraries.  Furthermore, the implementation enjoys a linear speed up with respect to the number of processors available.
	% !TEX root = Thesis.tex
\section{Relational Model}
	In its most basic form, the relational data model is built upon sets and tuples.  Each of these sets consist of a set of finite possible values.  Tuples are constructed from these sets to form relations.
	
	\begin{defn}[Named Tuple]
	\label{def:named-tuple}
		A named tuple \(\tuple\) is an instance of a relation \(\relation\), consisting of values corresponding to the attributes of \(\relation\).  For example,
	\end{defn}
	
	\begin{ex}
		Given a tuple \(\tuple = \{\text{code}: \text{``CDPS 101''}, \text{title}: \text{``Human-Mutant Relations''}, \text{subject}: \text{``CDPS''}\}\), we denote the attributes of \(\tuple\) as \(\attributes{\tuple} = \{\text{code}, \text{title}, \text{subject}\}\).  The values are \(\tuple\lbrack\text{code}\rbrack = \text{``CDPS 101''}\), \(\tuple\lbrack\text{title}\rbrack = \text{``Human-Mutant Relations''}\), and \(\tuple\lbrack\text{subject}\rbrack = \text{``CDPS''}\).
	\end{ex}
	
	\begin{defn}[Relation]
	\label{def:relation}
		A relation \(\relation\) is a set of named tuples, \(\relation = \{\tuple_1, \tuple_2, \dotsc, \tuple_n\}\), such that all the named tuples share the same attributes.
		\begin{equation}
			\forall \tuple, \tuple' \in \relation, \attributes{\tuple} = \attributes{\tuple'}
		\end{equation}
	\end{defn}
	
	\begin{ex}
		An example Course relation, \(\relation\), would be
		\[
			\relation = \left\{
				\begin{array}{llllll}
					\{\text{code} &: \text{``CDPS 101''}, &\text{title} &: \text{``Human-Mutant Relations''}, &\text{subject} &: \text{``CDPS''}\}, \\
					\{\text{code} &: \text{``CDPS 201''}, &\text{title} &: \text{``Humans and You''}, &\text{subject} &: \text{``CDPS''}\}, \\
					\{\text{code} &: \text{``MATH 360''}, &\text{title} &: \text{``Complex Analysis''}, &\text{subject} &: \text{``MATH''}\}
				\end{array}
			\right\}
		\]
	\end{ex}
		
	Relations are typically represented as tables.
	
	\begin{table}[!ht]
		\centering
		
		\begin{tabular}{lll}
			\toprule
			code & title & subject \\
			\midrule
			CDPS 101 & Human-Mutant Relations & CDPS \\
			CDPS 201 & Humans and You & CDPS \\
			MATH 360 & Complex Analysis & MATH \\
			\bottomrule
		\end{tabular}
		
		\caption{Course relation}
		\label{tbl:course-relation}
	\end{table}
	
	\begin{defn}[Keys]
	\label{def:keys}
		Keys are constraints imposed on relations.	A key constraint \(\key\) on a relation \(\relation\) is a subset of \(\attributes{\relation}\) which may uniquely identify a tuple.	Formally, we say \(\relation\) satisfies the key constraint \(\key\), denoted as \(\relation \models \key\), subject to
		\[
			\forall \tuple, \tuple' \in \relation, \tuple \not= \tuple' \implies \tuple\lbrack\key\rbrack \not= \tuple'\lbrack\key\rbrack
		\]
		
		For example, in \vref{tbl:course-relation}, the relation satisfies the key constraint \(\{\text{code}\}\) or \(\{\text{title}\}\), but not \(\{\text{subject}\}\).
	\end{defn}
	
	\begin{defn}[Foreign Keys]
	\label{def:foreign-keys}
		A \gls{fk} constraint applies to two relations, \(\relation_1, \relation_2\).  It asserts that values of certain attributes of \(\relation_1\) must appear as values of some corresponding attributes of \(\relation_2\).  A \gls{fk} constraint is written as
		\[
			\theta = \relation_1(\attribute_{11}, \attribute_{12}, \dotsc, \attribute_{1k}) \rightarrow \relation_2(\attribute_{21}, \attribute_{22}, \dotsc, \attribute_{2k})
		\]
		
		where \(\attribute_{1i} \subseteq \attributes{\relation_1}\) and \(\attribute_{2i} \subseteq \attributes{\relation_2}\).  We say \((\relation_1, \relation_2)\) satisfies \(\theta\), denoted as \((\relation_1, \relation_2) \models \theta\), if for every tuple \(\tuple \in \relation_1\), there exists a tuple \(\tuple' \in \relation_2\) such that \(\tuple\lbrack\attribute_{11}, \attribute_{12}, \dotsc, \attribute_{1k}\rbrack = \tuple'\lbrack\attribute_{21}, \attribute_{22}, \dotsc, \attribute_{2k}\rbrack\).
		
		We say \(\relation_1\) is the source, while \(\relation_2\) is the target.
				
		\begin{ex}
			Suppose we have a relation \(\rel{Course}{code, title, subject}\).	 We impose a \gls{fk} constraint of
			\begin{equation}
				\theta = \rel{Course}{subject} \rightarrow \rel{Subject}{id}
			\end{equation}
			
			which asserts \((\text{Course}, \text{Subject}) \models \theta\).  Therefore, if
			\[
				t = \{\text{code}: \text{``CDPS 101''}, \text{title}: \text{``Human-Mutant Relations''}, \text{subject}: \text{``CDPS''}\}
			\]
			
			then \(\exists! t' \in \text{Subject}\) such that \(t'\lbrack\text{id}\rbrack = \text{``CDPS''}\).
		\end{ex}
	\end{defn}
	
	\begin{defn}[Relational Database]
	\label{def:relational-database}
		A relational database, \(\db\), is a named collection of relations (as defined by \vref{def:relation}), keys (as defined by \vref{def:keys}), and foreign key constraints (as defined by \vref{def:foreign-keys}).
		
		We use \(\name{\db}\) to denote the name of \(\db\), \(\relations{\db}\) the list of relations in \(\db\), \(\keys{\db}\) the list of key constraints of \(\db\), and \(\fks{\db}\) the list of foreign key constraints of \(\db\).
	\end{defn}
	
	\subsection{Schema Group}
		\begin{defn}[Schema Graph]
		\label{def:schema-graph}
			If we view relations as vertices, and foreign key constraints as edges, a database \(\db\) can be viewed as a \emph{schema graph} \(\sgraph{}\), formally defined as
			\begin{align}
				\text{vertices} &: \text{V}(\sgraph) = \relations{\db} \\
				\text{edges} &: \text{E}(\sgraph) = \fks{\db}
			\end{align}
		\end{defn}
		
		\begin{ex}
			Given the schema in \vref{fig:schema}
			
			\begin{figure}[!ht]
				\centering
				
				\(\rel{Subject}{\underline{id}, name}\) \\
				\(\rel{Course}{\underline{code}, title, subject}\) \\
				\(\rel{Term}{\underline{id}, name}\) \\
				\(\rel{Section}{\underline{crn}, term, course}\) \\
				
				\caption{Subset of mycampus dataset schema}
				\label{fig:schema}
			\end{figure}
			
			and the \gls{fk} constraints in \vref{eqn:fk-constraints}
			
			\begin{align}\label{eqn:fk-constraints}
				\rel{Course}{subject} &\rightarrow \rel{Subject}{id} \\
				\rel{Section}{term} &\rightarrow \rel{Term}{id} \\
				\rel{Section}{course} &\rightarrow \rel{Course}{code}
			\end{align}
			
			we produce the schema graph in \vref{fig:schema-graph}
			
			\begin{figure}[!ht]
				\centering
				
				\begin{dot2tex}[dot]
					digraph G {
						node [shape=plaintext]; Subject; Course; Term; Section;
						
						Course -> Subject;
						Section -> Term
						Section -> Course;
					}
				\end{dot2tex}
				
				\caption{Graph representation of relations (\cref{fig:schema}) and \gls{fk} (\cref{eqn:fk-constraints}).}
				\label{fig:schema-graph}
			\end{figure}
		\end{ex}
		
		The relational data model is particularly powerful for analytic queries.  Given the schema graph in \vref{fig:schema-graph}, one can formulate the following analytic queries in a query language known as \gls{sql}.
			
		\begin{ex}
			Using \gls{sql}, find all section CRNs for the subject titled ``Community Development \& Policy Studies.''
			
			\begin{figure}[!ht]
				\begin{singlespaced}
					\begin{sqlcode}
SELECT section.crn
FROM   section 
       JOIN course 
         ON section.course_code = course.code 
       JOIN subject 
         ON subject.id = course.subject_id 
WHERE  subject.name = 'Community Development & Policy Studies'; 
					\end{sqlcode}
				\end{singlespaced}
				
				\caption{Query to find section CRNs for a subject name.}
				\label{fig:query-section-crns}
			\end{figure}
			
			The \gls{sql} query in \vref{fig:query-section-crns} results in \vref{tbl:query-section-crns-output}.
			
			\begin{table}[!ht]
				\centering
				
				\begin{tabular}{l}
					\toprule
					crn \\
					\midrule
					10000 \\
					10001 \\
					10002 \\
					\bottomrule
				\end{tabular}
				
				\caption{Results of the query in \vref{fig:query-section-crns}.}
				\label{tbl:query-section-crns-output}
			\end{table}
		\end{ex}
		
	\subsection{Entity Group}
		\begin{defn}[Entity Group]
		\label{def:entity-group}
			An entity group is a forest, \(\egraph\), of tuples interconnected by join conditions defined by the \gls{fk} constraints in the schema graph \(\sgraph\).
			
			Given two vertices \(\tuple, \tuple' \in \text{V}(\egraph)\), \(\exists \relation_1, \relation_2 \in \relations{\db}\) such that \(\tuple \in \relation_1\), \(\tuple'\in \relation_2\), and \((\relation_1, \relation_2)\in \sgraph\).  That is, \(\tuple\) and \(\tuple'\) belong to two relations that are connected by the schema graph.

			Let \(\relation_1(\attribute_{11}, \attribute_{12}, \dotsc, \attribute_{1k}) \to \relation_2(\attribute_{21}, \attribute_{22}, \dotsc, \attribute_{2k})\) be the \gls{fk} that connects \(\relation_1, \relation_2\).  We further assert that \(\tuple\lbrack\attribute_{11}, \attribute_{12}, \dotsc, \attribute_{1k}\rbrack = \tuple'\lbrack\attribute_{21}, \attribute_{22}, \dotsc, \attribute_{2k}\rbrack\).
		\end{defn}
		
		Entity groups define complex, structured objects that include more information than individual tuples in the relations.
		
		\begin{ex}
			The information in \vref{tbl:hmr-properties} all relates to the Course titled Human-Mutant Relations, however no single tuple in the database has all of this information as a result of database normalization.
			
			\begin{table}[!ht]
				\centering
				
				\begin{tabular}{ll}
					\toprule
					Attribute & Value \\
					\midrule
					code & CDPS 101 \\
					title & Human-Mutant Relations \\
					subject & Community Development \& Policy Studies \\
					\bottomrule
				\end{tabular}
				
				\caption{Properties of the Course titled Human-Mutant Relations.}
				\label{tbl:hmr-properties}
			\end{table}
			
			We require an entity group (\vref{fig:hmr-entity-group}) to join together all pieces of information related to this course. 
			
			\begin{figure}[!ht]
				\centering
				
				\begin{dot2tex}[dot]
digraph G {
	node [shape=plaintext]; "Human-Mutant Relations"; "Community Development & Policy Studies";
	
	"Human-Mutant Relations" -> "Community Development & Policy Studies";
}
				\end{dot2tex}
				
				\caption{Human-Mutant Relations entity group}
				\label{fig:hmr-entity-group}
			\end{figure}

		\end{ex}

	\subsection{Pros and Cons of the Relational Model}
		In order to better understand the motivation behind this work, it is important to examine both the strong and weak points of the relational model.
		
		\subsubsection{Pros}
			The enforcement of constraints is essential to the relational model.  There are several types of constraints, including uniqueness and \glspl{fk}.  The first constraint maintains uniqueness.
			
			The Course relation (\vref{tbl:course-relation}) has the attribute \texttt{code} as its primary key.	In order for other relations to reference a specific named tuple, the \texttt{code} attribute must be unique.
			
			\begin{ex}[Unique Constraint]
			\label{ex:unique-constraint}
				Attempt to insert another course with a \texttt{code} of ``CDPS 101.''
				
				\begin{singlespaced}
					\begin{sqlcode}
INSERT INTO course
VALUES      ('CDPS 101',
             'Mutant-Human Relations',
             'CDPS');
					\end{sqlcode}
				\end{singlespaced}
				
				The \gls{rdbms} enforces the primary key constraint on the \texttt{code} attribute, rejecting the insertion.
				
				\begin{verbatim}
Error: column code is not unique
				\end{verbatim}
			\end{ex}
			
			With the uniqueness of named tuples guaranteed (as demonstrated in \vref{ex:unique-constraint}), we must ensure that any named tuples that are referenced actually exist.  If they do not, the database must not permit the operation to continue.  Doing so would lead to dangling references.
			
			\begin{ex}[Referential Integrity]
				Attempt to insert the tuple (``CHEM 101'', ``Introductory Chemistry'', ``'CHEM'') in the Course relation.
				
				\begin{singlespaced}
					\begin{sqlcode}
INSERT INTO course
VALUES      ('CHEM 101',
             'Introductory Chemistry',
             'CHEM');
					\end{sqlcode}
				\end{singlespaced}
				
				Again we see the \gls{rdbms} protecting the integrity of the data.
				
				\begin{verbatim}
Error: foreign key constraint failed
				\end{verbatim}
			\end{ex}
			
			In addition to enforcing consistency, the relational model is capable of providing higher-level views of the data through aggregation.
			
			\begin{ex}[Aggregation]
				Find the number of sections offered for the subject named ``Community Development \& Policy Studies.''
				
				\begin{singlespaced}
					\begin{sqlcode}
SELECT Count(*)
FROM   section
       JOIN course
         ON section.course = course.code
       JOIN subject
         ON subject.id = course.subject
WHERE  subject.name = 'Community Development & Policy Studies';
					\end{sqlcode}
				\end{singlespaced}
			\end{ex}
			
			Information stored within a properly designed database is normalized.  That is, no information is repeated.
			
			\begin{ex}[Normalization]
				For example, suppose Emma Frost became headmistress and the subject named ``Community Development \& Policy Studies'' was renamed to ``Community Destruction \& Policy Studies.''  If this information were not normalized, each course in this subject would need to be updated.  Since this information is normalized, the following query will suffice.
				
				\begin{singlespaced}
					\begin{sqlcode}
UPDATE subject
SET    name = 'Community Destruction & Policy Studies'
WHERE  id = 'CDPS';
					\end{sqlcode}
				\end{singlespaced}
			\end{ex}
			
			The above examples are some of the most important reasons for choosing the relational model over others.	Unfortunately, the relational model is not without its downsides.
		
		\subsubsection{Cons}
			While the relational model excels at ensuring data consistency, aggregation, and reporting; it is not suitable for every task.	In order to issue queries, a user must be familiar with the schema.	 This requires specific domain knowledge of the data.
			
			An example of a complicated query involving two joins is give in \vref{fig:query-section-crns}.
			
			A casual user is unlikely to determine the correct join path, name of the tables, name of the attributes, etc.	This is in contrast to the document model, where the data is semi-structured or unstructured, requiring minimal domain knowledge.

			The relational model is also rigid in structure.  If a relation is modified, every query referencing said relation may require a rewrite.  Even a simple attribute being renamed (e.g.~\(\rho_{\text{name/alias}}(\text{Person})\)) is capable of modifying the join paths.  This rigidity places additional cognitive burden on users.
			
			In addition to having a rigid structure, most relational database management systems lack flexible string matching options.	 Assuming basic SQL-92 compliance, a \gls{rdbms} only supports the \texttt{LIKE} predicate \cite{sql-2011}.
			
			\begin{ex}[\texttt{LIKE} Predicate]
				Find all courses with a title that contains ``man.''
				
				\begin{singlespaced}
					\begin{sqlcode}
SELECT *
FROM   course
WHERE  title LIKE '%man%';
					\end{sqlcode}
				\end{singlespaced}
			\end{ex}
			
			There are a couple of limitations to the \texttt{LIKE} predicate.  First, it only supports basic substring matching.  If a user accidentally searches for all courses with a title containing ``men,'' nothing would be found.
			
			Second, unless the predicate is applied to the end of the string and the column is indexed, performance will be poor.  The database must scan the entire table in order to answer the query, resulting in performance of \(\mathcal{O}(n)\), where \(n\) is the number of named tuples in the relation.
	% !TEX root = Thesis.tex
\section{Document Model}
\label{sec:document-model}
	In this section we formally define the document model.
	
	Documents are a unit of information.  The definition of unit can vary.  It may represent an email, a book chapter, a memo, etc.  Contained within each document is a set of terms.
	
	In contrast to the relational model, the document model represents semi-structured as well as unstructured data.  Examples of information suitable to the document model includes emails, memos, book chapters, etc.
	
	These pieces, or units, of information are broken into documents.  Groups of related documents (for example, a library catalogue) are referred to as a document collection.

	\begin{defn}[Terms and Document]
	\label{def:document}
		A term, $t$, is an indivisible string (e.g.~a proper noun, word, or a phrase).  A document, $d$, is a bag of words.  Let $\freq\left(t, d\right)$ be the frequency of terms $t$ in document $d$.
		
		Let $T$ denote all possible terms, and $\fcn{Bag}{T}$ be all possible bag of terms.
	\end{defn}
	
	\begin{remark}
		We use the bag-of-words model for documents.  This means that position information of terms in a document is irrelevant, but the frequency of terms are kept in the document.  Documents are non-distinct sets.
	\end{remark}
	
	\begin{defn}[Document Collection]
	\label{def:document-collection}
		A document collection $D$ is a set of documents, written $D = \left\{d_1, d_2, \dotsc, d_k\right\}$.  The size of $D$ is denoted $\gls{ndocs}$.  The number of unique terms, or size of $\gls{terms}$, in $D$, is denoted $\gls{nterms}$.
	\end{defn}
	
	\begin{ex}
	\label{ex:superhero-documents}
		Consider the following short sentences.
		
		\begin{enumerate}
			\item Superman is strong on Earth and lives on Earth.
			\item Batman was born on Earth.
			\item Superwoman is fast on Earth.
			\item Superman was born on Krypton.
		\end{enumerate}
		
		Each sentence represents a document, giving us the following documents.
		
		\begin{eqnarray*}
			d_1 &=& \left\{\textrm{``and''}: 1, \textrm{``on''}: 2, \textrm{``is''}: 1, \textrm{``lives''}: 1, \textrm{``earth''}: 2, \textrm{``strong''}: 1, \textrm{``superman''}: 1\right\} \\
			d_2 &=& \left\{\textrm{``batman''}: 1, \textrm{``on''}: 1, \textrm{``was''}: 1, \textrm{``earth''}: 1, \textrm{``born''}: 1\right\} \\
			d_3 &=& \left\{\textrm{``on''}: 1, \textrm{``is''}: 1, \textrm{``superwoman''}: 1, \textrm{``fast''}: 1, \textrm{``earth''}: 1\right\} \\
			d_4 &=& \left\{\textrm{``krypton''}: 1, \textrm{``born''}: 1, \textrm{``on''}: 1, \textrm{``was''}: 1, \textrm{``superman''}: 1\right\} \\
		\end{eqnarray*}
	\end{ex}
	
	\subsection{Vectorization of Documents}
	\label{sec:vectorization-of-documents}
		One of the most fundamental approach for search documents is to treat documents as high dimensional vectors, and the document collection as a subset in a vector space.  The search query becomes a nearest neighbour query in a vector space equipped with a distance measure.
		
		The first step is to convert bag of terms into vectors.  The standard technique \cite{ir-08} uses a scoring function that measures the relative importance terms in documents.
		
		\begin{defn}[TF-IDF Score]
			The term frequency is the number of times a term $t$ appears in a document $d$, as given by $\freq\left(t, d\right)$.  The document frequency of a term $t$, denoted by $\df_t$, is the number of documents in $D$ that contains $t$.  It is defined as
			
			$$\df_t = \mid \left\{d \in D: t \in d\right\} \mid$$
			
			The combined TF-IDF score of $t$ in a document $d$ is given by
			
			$$\tfidf\left(D, t, d\right) = \frac{\freq\left(t, d\right)}{\mid d \mid} \cdot \log{\frac{N}{\df_t}}$$
		\end{defn}
		
		\begin{remark}
			The first component, $\frac{\freq\left(t, d\right)}{\mid d \mid}$, measures the importance of a term within a document.  It is normalized to account for document length.  The second component, $\log{\frac{N}{\df_t}}$, is a measure of the rarity of the term within the document collection $D$.
		\end{remark}
		
		\begin{ex}
			Using the documents from Example~\ref{ex:superhero-documents}, the TF-IDF scores are as follows.
			
			$$\bordermatrix{
				~ & d_1 & d_2 & d_3 & d_4 \cr
				t_1 : \textrm{``and''} & 0.2857 & 0.0000 & 0.0000 & 0.0000 \cr
				t_2 : \textrm{``on''} & 0.0000 & 0.0000 & 0.0000 & 0.0000 \cr
				t_3 : \textrm{``superwoman''} & 0.0000 & 0.0000 & 0.4000 & 0.0000 \cr
				t_4 : \textrm{``batman''} & 0.0000 & 0.4000 & 0.0000 & 0.0000 \cr
				t_5 : \textrm{``is''} & 0.1429 & 0.0000 & 0.2000 & 0.0000 \cr
				t_6 : \textrm{``fast''} & 0.0000 & 0.0000 & 0.4000 & 0.0000 \cr
				t_7 : \textrm{``born''} & 0.0000 & 0.2000 & 0.0000 & 0.2000 \cr
				t_8 : \textrm{``krypton''} & 0.0000 & 0.0000 & 0.0000 & 0.4000 \cr
				t_9 : \textrm{``earth''} & 0.1186 & 0.0830 & 0.0830 & 0.0000 \cr
				t_{10} : \textrm{``lives''} & 0.2857 & 0.0000 & 0.0000 & 0.0000 \cr
				t_{11} : \textrm{``strong''} & 0.2857 & 0.0000 & 0.0000 & 0.0000 \cr
				t_{12} : \textrm{``was''} & 0.0000 & 0.2000 & 0.0000 & 0.2000 \cr
				t_{13} : \textrm{``superman''} & 0.1429 & 0.0000 & 0.0000 & 0.2000 \cr
			}$$
		\end{ex}

		\begin{defn}[Document Vector]
			Given a document collection $D$ with $M$ unique terms $T = \left[ t_1, t_2, \dotsc, t_n \right]$, each document $d$ can be represented by an $M$-dimensional vector.
			
			$$
				\vec{d} = 
				\left[
				\begin{array}{c}
					\tfidf\left(t_1, d\right) \\
					\tfidf\left(t_2, d\right) \\
					\vdots \\
					\tfidf\left(t_n, d\right)
				\end{array}
				\right]
			$$
		\end{defn}
		
		\begin{ex}
			The documents in Example~\ref{ex:superhero-documents} would produce the following vectors.
			
			$$
			\vec{d_n} = 
				\left[
					\begin{array}{l}
						\tfidf\left(t_{1}, d_n\right) \\
						\tfidf\left(t_{2}, d_n\right) \\
						\tfidf\left(t_{3}, d_n\right) \\
						\tfidf\left(t_{4}, d_n\right) \\
						\tfidf\left(t_{5}, d_n\right) \\
						\tfidf\left(t_{6}, d_n\right) \\
						\tfidf\left(t_{7}, d_n\right) \\
						\tfidf\left(t_{8}, d_n\right) \\
						\tfidf\left(t_{9}, d_n\right) \\
						\tfidf\left(t_{10}, d_n\right) \\
						\tfidf\left(t_{11}, d_n\right) \\
						\tfidf\left(t_{12}, d_n\right) \\
						\tfidf\left(t_{13}, d_n\right) \\
					\end{array}
				\right],
			\vec{d_1} = 
				\left[
					\begin{array}{l}
						0.2857 \\
						0.0000 \\
						0.0000 \\
						0.0000 \\
						0.1429 \\
						0.0000 \\
						0.0000 \\
						0.0000 \\
						0.1186 \\
						0.2857 \\
						0.2857 \\
						0.0000 \\
						0.1429 \\
					\end{array}
				\right],
			\vec{d_2} = 
				\left[
					\begin{array}{l}
						0.0000 \\
						0.0000 \\
						0.0000 \\
						0.4000 \\
						0.0000 \\
						0.0000 \\
						0.2000 \\
						0.0000 \\
						0.0830 \\
						0.0000 \\
						0.0000 \\
						0.2000 \\
						0.0000 \\
					\end{array}
				\right],
			\vec{d_3} = 
				\left[
					\begin{array}{l}
						0.0000 \\
						0.0000 \\
						0.4000 \\
						0.0000 \\
						0.2000 \\
						0.4000 \\
						0.0000 \\
						0.0000 \\
						0.0830 \\
						0.0000 \\
						0.0000 \\
						0.0000 \\
						0.0000 \\
					\end{array}
				\right],
			\vec{d_4} = 
				\left[
					\begin{array}{l}
						0.0000 \\
						0.0000 \\
						0.0000 \\
						0.0000 \\
						0.0000 \\
						0.0000 \\
						0.2000 \\
						0.4000 \\
						0.0000 \\
						0.0000 \\
						0.0000 \\
						0.2000 \\
						0.2000 \\
					\end{array}
				\right]
			$$
		\end{ex}
		
		\begin{defn}[Search Query]
			A search query $q$ is simply a document, namely a bag of terms.  The top-$k$ answers to $q$ with respect to a collection $D$ is defined as the $k$ documents, $\left\{d_1, d_2, \dotsc, d_k\right\}$, in $D$, such that $\left\{\vec{d}_i\right\}$ are the closest vectors to $\vec{q}$ using Euclidean distance measure in $\mathbb{R}^N$.
		\end{defn}
		
		\begin{ex}
			Given the search query $q = \left\{ \mathrm{superwoman}, \mathrm{was}, \mathrm{born}, \mathrm{on}, \mathrm{krypton} \right\}$, compute the vector $\vec{q}$ within the document collection $D$ (as defined in Example~\ref{ex:superhero-documents}).
			
			$$
			\vec{q} = 
				\left[
					\begin{array}{c}
						0.0000 \\
						0.0000 \\
						0.0000 \\
						0.0000 \\
						0.0000 \\
						0.0000 \\
						0.0000 \\
						0.1474 \\
						0.2644 \\
						0.0000 \\
						0.2644 \\
						0.1474 \\
						0.0000 \\
					\end{array}
				\right]
			$$
		\end{ex}
		
		In order to determine the top-$k$ documents for search query $q$, we need a way of measuring the similarity between documents.
		
		\begin{defn}[Cosine Similarity]
			Given two document vectors, $\vec{d}_1$ and $\vec{d}_2$, the cosine similarity is the dot product $\vec{d}_1 \cdot \vec{d}_2$, normalized by the product of the Euclidean distance of $\vec{d}_1$ and $\vec{d}_2$ in $\mathbb{R}^N$.  It is denoted as $\similarity\left(\vec{d}_1, \vec{d}_2\right)$.
			
			\begin{eqnarray}
				\similarity\left(\vec{d}_1, \vec{d}_2\right) &=& \frac{\vec{d}_1 \cdot \vec{d}_2}{\mid\mid \vec{d}_1 \mid\mid \cdot \mid\mid \vec{d}_2 \mid\mid} \\
				 &=& \frac{\sum\limits_{i=1}^{N} \vec{d}_{1, i} \times \vec{d}_{2, i}}{\sqrt{\sum\limits_{i=1}^{N} \left(\vec{d}_{1, i}\right)^2} \times \sqrt{\sum\limits_{i=1}^{N} \left(\vec{d}_{2, i}\right)^2}}
			\end{eqnarray}
		\end{defn}
		
		Recall we may represent search queries as documents and thus document vectors.  Therefore we may compute the score of a document $d$ for a search query $q$ as
		
		$$\similarity\left(\vec{d}, \vec{q}\right)$$
		
		\begin{ex}
			Given the document collection $D$ (from Example~\ref{ex:superhero-documents}) and search query $q$, compute the similarity between $q$ and every document $d \in D$.
			
			\begin{eqnarray}
				\similarity\left(\vec{d}_1, \vec{q}\right) &=& 0.000000 \\
				\similarity\left(\vec{d}_2, \vec{q}\right) &=& 0.191533 \\
				\similarity\left(\vec{d}_3, \vec{q}\right) &=& 0.265877 \\
				\similarity\left(\vec{d}_4, \vec{q}\right) &=& 0.618553
			\end{eqnarray}
		\end{ex}
		
	\subsection{Extending the Document Model}
	\label{sec:extending-the-document-model}
		In the extended document model, documents have attributes: $\fcn{ATTR}{d}$, and each attribute have values (e.g.~date, string, integer), or bag of terms.  Thus:

		$$d:\fcn{ATTR}{d} \to \fcn{BAG}{\mathrm{Terms}}$$
		
		\begin{ex}[Semi-Structured Document]
			We see that $d_2$ is about Batman.  The document contents are semi-structured, containing both a name and the name of a planet.  By adding attributes to the document, we are left with Table~\ref{tbl:person-document}.
			
			\begin{table}[!ht]
				\centering
				
				\begin{tabular}{ll}
					\toprule
					Attribute & Value \\
					\midrule
					name & Batman \\
					birthplace & Earth \\
					body & Batman was born on Earth. \\
					\bottomrule
				\end{tabular}
				
				\caption{Person document for Batman}
				\label{tbl:person-document}
			\end{table}
			
			which is similar in structure to the \texttt{Person} table.
		\end{ex}
		
	\subsection{Approximate String matching}
	\label{sec:n-gram}
		\begin{defn}[N-Gram]
			An $n$-Gram is a contiguous sequence of substrings of string $S$ of length $n$.  An algorithm for computing the $n$-gram of $S$ is given in Algorithm~\ref{alg:n-gram}. 
		\end{defn}
		
		% \char"24 - DOLLAR  BILL Y'ALL

		\begin{algorithm}[!ht]
			\caption{$\textsc{N-Gram}\left(S, n, s\right)$}
			\label{alg:n-gram}
			
			%\begin{singlespaced}
				\begin{algorithmic}[1]
					\REQUIRE $S$ is a string, $n \ge 1$, and $s$ is a character
					\ENSURE the list of $n$-grams of $S$
					\medskip
					\STATE $G \leftarrow \left[\right]$\label{alg:n-gram:G}
					\STATE $p \leftarrow \textsc{Repeat}\left(s, n - 1\right)$
					\STATE $S \leftarrow \textsc{Pad}\left(S, p\right)$
					\STATE $S \leftarrow \textsc{Replace}\left(S, `\;', p\right)$\label{alg:n-gram:S}
					
					\FOR{$i=0$ \TO $l - n + 1$}
						\STATE append $S\left[i, i + n\right]$ to $G$
					\ENDFOR
					
					\RETURN $G$
			%		\medskip
			%		\medskip
				\end{algorithmic}
			%\end{singlespaced}
		\end{algorithm}
		
		%\todo{The medskips above should not be required, but are in single spaced mode.}
		
		Where $l$ is the length of $S$, $\textsc{Repeat}\left(S, n\right)$ repeats $s$ character $n$ times, $\textsc{Pad}\left(S, p\right)$ prefixes and postfixes $S$ with $p$, and $\textsc{Replace}\left(S, s, p\right)$ replaces character $s$ with $p$ in string $S$.
		
		\begin{ex}
		\label{ex:ngram-banana}
			Given a string $S = \mathrm{``superman''}$, compute the trigram of $S$ using Algorithm \ref{alg:n-gram}.  Lines \ref{alg:n-gram:G}-\ref{alg:n-gram:S} would yield $S = \mathrm{``\char"24\char"24superman\char"24\char"24''}$, resulting in $l = 12$.
			
			$$
				G = \left\{
					\mathrm{``\char"24\char"24s''},
					\mathrm{``\char"24su''},
					\mathrm{``sup''},
					\mathrm{``upe''},
					\mathrm{``per''},
					\mathrm{``erm''},
					\mathrm{``rma''},
					\mathrm{``man''},
					\mathrm{``an\char"24''},
					\mathrm{``n\char"24\char"24''}
				\right\}
			$$
		\end{ex}
		
		We use $n$-grams in order to permit approximate string matching.
		
		\begin{ex}
		\label{ex:n-gram-comparison}
			Given a string $S$ (Example~\ref{ex:ngram-banana}), let $S' = \mathrm{``superwoman''}$.  Compute the trigram of $S'$ and compare it to $S$.
			
			$$
				G' = \left\{
					\mathrm{``\char"24\char"24s''},
					\mathrm{``\char"24su''},
					\mathrm{``sup''},
					\mathrm{``upe''},
					\mathrm{``per''},
					\mathrm{``erw''},
					\mathrm{``rwo''},
					\mathrm{``wom''},
					\mathrm{``oma''},
					\mathrm{``man''},
					\mathrm{``an\char"24''},
					\mathrm{``n\char"24\char"24''}
				\right\}
			$$
			
			Comparing $G$ to $G'$ results in the following matrix
			
			\begin{figure}[!ht]
				$$
					\bordermatrix{
						~ & G & G' \cr
						t_1 : \mathrm{``an\char"24''} & 1 & 1 \cr
						t_2 : \mathrm{``oma''} & 0 & 1 \cr
						t_3 : \mathrm{``\char"24su''} & 1 & 1 \cr
						t_4 : \mathrm{``rwo''} & 0 & 1 \cr
						t_5 : \mathrm{``rma''} & 1 & 0 \cr
						t_6 : \mathrm{``man''} & 1 & 1 \cr
						t_7 : \mathrm{``erw''} & 0 & 1 \cr
						t_8 : \mathrm{``\char"24\char"24s''} & 1 & 1 \cr
						t_9 : \mathrm{``upe''} & 1 & 1 \cr
						t_{10} : \mathrm{``n\char"24\char"24''} & 1 & 1 \cr
						t_{11} : \mathrm{``per''} & 1 & 1 \cr
						t_{12} : \mathrm{``wom''} & 0 & 1 \cr
						t_{13} : \mathrm{``sup''} & 1 & 1 \cr
						t_{14} : \mathrm{``erm''} & 1 & 0 \cr
					}
				$$
				
				\caption{Comparison between $n$-grams of $G$ and $G'$.}
				\label{fig:n-gram-misspelling-comparison}
			\end{figure}
			
			As Figure~\ref{fig:n-gram-misspelling-comparison} shows, using $n$-grams yield a similarity of $\frac{8}{14}$.
		\end{ex}
			
	\subsection{Pros and Cons of the Document Model}
		There are numerous reasons to use the document model.  It allows users without domain knowledge and working knowledge of a complex query language such as \gls{sql} to find information.
		
		\begin{ex}[Simple Queries]
			Find all documents related to ``Superman'' or ``Earth''.  This query, if the default operator is \texttt{OR}, would simply be \texttt{Superman Earth}.  The result of the query $q$ would be
			
			$$\query\left(\mathrm{``superman''}\right) \cup \query\left(\mathrm{``earth''}\right) \rightarrow \left\{d_1, d_2, d_3, d_4\right\}$$
		\end{ex}
		
		Users can also modify queries to require certain terms be present or not present.
		
		\begin{ex}[\texttt{AND} Query]
		\label{ex:and-query}
			Find all documents containing both ``Superman'' and ``Earth''.  This query would return the following set of documents
			
			$$\query\left(\mathrm{``superman''}\right) \cap \query\left(\mathrm{``earth''}\right) \rightarrow \left\{d_1\right\}$$
			
			as only $d_1$ contains both terms.
		\end{ex}
		
		\begin{ex}[\texttt{NOT} Query]
			Find all documents containing ``Superman'' but not ``Earth''.  This query would return different results than Example~\ref{ex:and-query}.
			
			$$\query\left(\mathrm{``superman''}\right) \neg\query\left(\mathrm{``earth''}\right) \rightarrow \left\{d_4\right\}$$
		\end{ex}
		
		While none of the above queries required domain knowledge, it is possible to use the extended document model (Section~\ref{sec:extending-the-document-model}) to search specific fields.  Doing so allows users to have finer control over what documents are retrieved.
		
		\begin{ex}[Extended Query]
			Find all documents with a superhero named ``Superman'' that contain the term ``Earth''.
			
			$$\query\left(\mathrm{``name''}, \mathrm{``superman''}\right) \cap \query\left(\mathrm{``earth''}\right) \rightarrow \left\{d_1\right\}$$
			
			Assuming the first term of every document is also the value of the name attribute.
		\end{ex}
		
		People utilize keyword query search every day through web search engines such as Google\footnote{\url{https://www.google.ca/}}.
		
		Not only does the document model provide a familiar interface to search for information with, it also ranks the results.  In the relational model a search for ``Superman'' would return all named tuples that contained that term.  In the document model, documents are ranked against the query $q$ and the top-$k$ documents are returned.
		
		The advantage is that users have the result of $q$ already ranked so only the most relevant documents may be explored.  As the number of documents matching $q$ for a large corpus can be high, showing only the top-$k$ relevant documents may save the user a substantial amount of time.
		
		The relational model does not permit approximate string matching.  By utilizing the document model with $n$-grams (Section~\ref{sec:n-gram}), users who substitute, delete, or insert characters from the desired term may still receive results for their intended term (see Example~\ref{ex:n-gram-comparison} for a demonstration of how $n$-grams overcome character substitutions).
		
		Unfortunately the document model does not support the concept of foreign keys (Definition~\ref{def:foreign-keys}).  While information is easily accessible due to flexible search, each document is a discrete unit of information.  Aggregate queries are unsupported, as these units are not linked amongst one another.
	
	\section{Problem}
		Each of these data models has its own pros and cons.
		
		\begin{itemize}
			\item The relational model is built upon the enforcement of constraints.  These constraints exist to protect the integrity of the data.
			\item The document model allows users to use simple keyword queries and a small set of operators to quickly locate data with minimal knowledge of its structure.
			\item The relational model is rigid and data is highly normalized, while the document model is flexible and de-normalized.
		\end{itemize}
		
		One must choose between highly normalized, structured data and fast, flexible keyword search.
		
		What is needed is a system that is capable of automatically transforming data from the relational model to the document model.  This would provide the benefits of both models.  Such a system would require some initial configuration, but would require little user intervention afterwards.
	
	\section{Thesis Statement \& Scope of Research}
		\begin{displayquote}
			\textbf{Thesis Statement:}  A system could be built that is capable of transforming data from the relational model to the document model.  The transformation is reversible, allowing the original data model to be recovered.  This system would use the keyword search capabilities, along with the relational information, to quickly discover related fragments of information.
		\end{displayquote}
		
		In order to achieve the goal of our thesis statement, we must
		
		\begin{itemize}
			\item Define a formal framework to describe data sets with relational structures and text components,
			\item Design a collection of expressive query operators for analyzing text relational data sets; and
			\item Investigate implementation techniques to make the query operators performant on modern, multicore computers.
		\end{itemize}
		
	 \section{Contributions}
	 	We provide a formal definition for a system that is capable of transforming data from the relational model to the document model.  By performing this transformation, we gain the flexible search characteristics of the document model.
	 	
	 	This transformation is done in such a way that it is reversible.  In order to be reversible, the relational information is encoded in a document form.  This allows us to perform graph search over documents.
	 	
	 	In addition, we investigate the effect a concurrent graph search implementation has on the performance.  We believe that the rate of growth of the search time will be reduced by performing the graph search concurrently.
	
	\section{Outline of Thesis}
		\cref{chap:best-of-both-worlds} presents a framework for the transformation between the two document models.  It demonstrates how to encode named tuples as documents.  It further describes a method of encoding relational data in the document data model, permitting iterative graph search.
		
		\cref{chap:along-came-clojure} describes the details of our implementation the data transformation and query operators for graph search in the linked document space.  Our choice of utilizing a modern functional programming language for our implementation makes high degree of concurrency possible.
		
		\cref{chap:experimental-evaluation} provides further justification to our choice of data model mapping and the choice of programming language used.  Through a series of experiments, we see that our proposal allows a tight integration of the relational database engines and keyword search libraries.  Furthermore, the implementation enjoys a linear speed up with respect to the number of processors available.
		
		Finally, in \cref{chap:conclusion} we present a summary of our findings.  We highlight limitations of the work and suggest topics for future research.  We provide an account of the various lessons learned while performing this research and building the system.
	
	\chapter{Best of Both Worlds}
	\section{Encoding Named Tuples into Documents}
	\label{sec:named-tuples-documents}
		Recall in the extended document model (\vref{sec:extending-the-document-model}), a document \(\doc\) consists of fields \(\field_1, \field_2, \dotsc, \field_n\).  Using the extended document model, we are left with a straight forward mapping of a tuple \(\tuple\) to document \(\doc\).
		
		For tuple \(\tuple\), every attribute \(\attribute \in \attributes{\tuple}\) maps to field \(\field\) in \(\doc\).	Every attribute value must be analyzed into an indexable form in order to store it in a field.
		
		\begin{align}
			\attributes{\tuple} &\xrightarrow{analyzed} \fields{\doc} \\
			\attribute_1, \attribute_2, \dotsc, \attribute_n &\xrightarrow{analyzed} \field_1, \field_2, \dotsc, \field_n
		\end{align}
		
		We denote the document encoding of \(\tuple\) as \(\docs{\tuple}\).
	
	\section{Mapping of Entity Groups to Documents}
		Recall that an entity group (\vref{def:entity-group}) is a forest \(\egraph\) of tuples \(\tuple\) such that
		
		\[
			\forall (\tuple, \tuple') \in \egraph, \tuple \nequal \tuple' \Rightarrow \relations{\tuple} \nequal \relations{\tuple'}
		\]
		
		That is, all tuples are from distinct relations.
		
		Given the restriction
		
		\[
			\forall (\relation, \relation') \in \sgraph{\db}, \exists! (\relation, \relation') \models \theta
		\]
		
		we assert that if \((\tuple, \tuple') \in \egraph\), then \((\tuple, \tuple') \in \relations{\tuple} \bowtie \relations{\tuple'}\).
		
		Let \(\text{V}(\egraph)\) be the vertices of \(\egraph\), \(\text{E}(\egraph)\) be the edges of \(\egraph\).
		
		\begin{claim}
		\label{clm:lossless}
			\(\egraph\) can always be reconstructed from \(\text{V}(\egraph)\) without loss of information.
		\end{claim}
		
		\begin{proof}
			Given \(\text{V}(\egraph)\), we must reconstruct \(\text{E}(\egraph)\) in order to complete \(\egraph\).
			
			Choose any \((\tuple, \tuple') \in \text{V}(\egraph)\).	If \((\relations{\tuple}, \relations{\tuple'}) \in \sgraph{\db}\), then \((\tuple, \tuple')\) is an edge in \(\egraph\).
			
			Recall our earlier assertion that \(\sgraph{\db}\) is cycle-free and foreign keys must be unique.
		\end{proof}
		
	\section{Encoding an Entity Group as a Document Group}
		Given a entity group \(\egraph\), we construct two or more documents in order to represent the entity group in the document model.
		
		For every \(\tuple \in \text{V}(\egraph)\), we construct a document \(\docs{\tuple}\) (\vref{sec:named-tuples-documents}).  With each tuple \(\tuple\) stored in the document collection \(\dc\), we construct an additional document which stores the association information.
		
		Let \(x\) be the indexing document of \(\egraph\).
		
		\[
			x[\text{``entities''}] = \bigcup_{\tuple \in \text{V}\left(\egraph\right)} \uid{\tuple}
		\]
		
		Thus, the encoding of \(\egraph\) is defined as
		
		\[
			\egraph \xrightarrow{\text{encode}} \{\docs{\tuple} : \tuple \in V(\egraph)\} \cup \{x\}
		\]
		
		It's easy to see that from \(\text{encode}(\egraph)\) we can recover \(\text{V}(\egraph)\), the tuples in \(\egraph\).
		
		By \vref{clm:lossless}, this is sufficient to recover \(\egraph\) entirely.
	
	\section{Encoding Attribute Values into Searchable Documents}
		Each value for user selected attributes are converted into \(n\)-grams, and stored in special documents.
	
	\section{Iterative Search Using Document Encodings}
		A document database supports fast and flexible keyword search queries.	A search query is characterized by \(q = (f, w)\), where \(f\) is a field name, and \(w\) is a search phrase.
		
		Search[q] is the set of documents returned by the text index.  The search function allows us to do many things:
		
		\begin{enumerate}
			\item Suggest values, correcting spelling errors
			\item Given attribute values, search all relevant entities
			\item Search for all relevant entity groups of one or more entities, using the indexing document
			\item We can connect entities via entity groups using (hyper)graph search algorithms.
		\end{enumerate}
	
	\chapter{Along Came Clojure}
\label{sec:along-came-clojure}
	Talk about how great Clojure is.
	
	\section{Basic principles of functional programming (2 days)}
	\begin{itemize}
		\item immutable data structures
		\item persistent data structures using multi-versioning
		\item functions (and higher order functions) as values
	\end{itemize}

	\subsection{Features of Clojure (2 days)}
		\begin{itemize}
			\item Data structures supporting the universal design pattern
			\item Concurrency + STM
			\item Interoperability with JVM (including Lucene)
		\end{itemize}
	\section{Search With Clojure}
\label{sec:search-with-clojure}
	Clojure's first-class \gls{jvm} interoperability permits the use of countless third-party libraries.  The most extensively used was Lucene.
	
	\subsection{Full-Text Search Using Lucene}
		\textcquote{luc-home}{Apache Lucene\texttrademark\ is a high-performance, full-featured text search engine library written entirely in Java.}  Lucene implements the document model, and provides a simple yet powerful and feature-rich \gls{api} to perform full-text search.  Among these features is the ability to vectorize documents according to the vector space model, utilize the extended document model to provide search of semi-structured documents, and issue search queries against the index.
	
	\subsection{Indexing Relational Database}
		The indexing of relational objects is a complicated process.  The objects must be retrieved from the relational database, transformed into documents from named tuples, then placed in the index.  Additionally, all foreign keys (\cref{def:foreign-keys}) must be encoded as documents (\cref{sec:encoding-entity-group-as-document-group}) to satisfy \cref{sec:mapping-entity-groups-to-documents}.
		
		The schema graph (\cref{def:schema-graph}) must be defined before the relational database may be crawled.
		
		\subsubsection{Schema Graph Definition}
		\label{sec:entity-schema}
			The schema graph is defined using a list.  Each schema component, whether an entity or entity group, is defined by \texttt{EntitySchema} records.  Each record accepts a map which specifies how each class of document should be indexed and identified.  The keys of this map are given in \cref{tbl:entity-schema-keys}.
			
			\begin{table}
				\centering
				
				\begin{tabular}{lp{9cm}l}
					\toprule
					Key & Description & Type(s) \\
					\midrule
					\texttt{:T} & Entity (\texttt{:entity}) or entity group (\texttt{:entity}) & Symbol \\
					\texttt{:C} & Table name for entities, brief description for entity groups & Symbol or String \\
					\texttt{:sql} & \Gls{sql} query used to construct the entity or entity schema & Expression \\
					\texttt{:ID} & Attribute or attributes that comprise the key (\cref{def:keys}) & Symbol or list of symbols \\
					\texttt{:attrs} & List of attributes to analyze to fields & List of symbols \\
					\texttt{:values} & List of attributes to index as values, must be subset of \texttt{:attrs} & List of symbols \\
					\bottomrule
				\end{tabular}
				
				\caption{Keys expected by \texttt{EntitySchema} records}
				\label{tbl:entity-schema-keys}
			\end{table}
		
		\subsubsection{Crawling the Relational Database}
			With the schema graph defined, the system is able to crawl the relational database, yielding a sequence of named tuples.  It iterates through every \texttt{EntitySchema} record, instructing every record to crawl itself given a database connection and index writer (\cref{clj:building-the-index}).
			
			\begin{figure}
				\begin{singlespaced}
					\begin{pygments}{clj}
(doseq [ent-def schemas]
  (crawl ent-def db-conn idx-w))
					\end{pygments}
				\end{singlespaced}
				
				\caption{Building the index}
				\label{clj:building-the-index}
			\end{figure}
			
			The \texttt{Database} protocol provides an \texttt{execute-query} method that permits access to the database.  In the current implementation, the \texttt{Sqlite3} record implements the \texttt{Database} protocol.  This record issues the query as-is, applying a given function to every result.
			
			Each record issues a \gls{sql} query against the database that retrieves all named tuples that it represents.  This query is given by the \texttt{:sql} key of the record.  For every symbol defined by the \texttt{:values} key, an additional query is issued.  These queries retrieve all distinct (within the context of that relation and attribute) values.
		
		\subsubsection{Transformation}
			For every named tuple, a document is constructed.  In addition to the attributes, several other fields may be added to the document.  These special fields contain additional meta information about the document.  For example, the class, type, and unique identifier are added to an entity, while an entity group has a space-delimited list of unique identifiers that comprise the group.
			
			Before becoming a document, named tuples are transformed into an internal representation.  The internal representation adds flexibility to the system; so long as functions exist to convert between the internal representation and other forms, the source of the data is irrelevant.
			
			Clojure permits the annotation of data with metadata (\cref{clj:with-meta-internal-rep}).  Named tuples are returned as maps, with key-value pairs representing attributes and values for the tuple.  Metadata may be associated with a named tuple.  The metadata does not affect its key-value pairs.
			
			The map of attributes and values of a named tuple is annotated with the function \texttt{(with-meta obj map)}.  The \texttt{obj} parameter is the named tuple, while \texttt{map} is a map of metadata as defined by the system.  The keys of \texttt{map} for each type (value, entity, or entity group) are defined in \cref{tbl:type-metadata}.  The internal representation of the named tuple given in \cref{tbl:hmr-properties-rel} is given in \cref{clj:with-meta-internal-rep}.
			
			\begin{table}
				\begin{subtable}[b]{0.33\linewidth}
					\centering
					
					\begin{tabular}{ll}
						\toprule
						Key & Value \\
						\midrule
						:type & :value \\
						:class & <rel>|<attr> \\
						\bottomrule
					\end{tabular}
					
					\caption{Value}
				\end{subtable}
				\begin{subtable}[b]{0.33\linewidth}
					\centering
					
					\begin{tabular}{ll}
						\toprule
						Key & Value \\
						\midrule
						:type & :entity \\
						:class & <rel> \\
						:ID & <rel>|<pk> \\
						\bottomrule
					\end{tabular}
					
					\caption{Entity}
				\end{subtable}
				\begin{subtable}[b]{0.33\linewidth}
					\centering
					
					\begin{tabular}{ll}
						\toprule
						Key & Value \\
						\midrule
						:type & :group \\
						:entities & <rel>|<pk> \\
						 & [<rel>|<pk> \ldots] \\
						\bottomrule
					\end{tabular}
					
					\caption{Entity Group}
				\end{subtable}
				
				\caption{Metadata associated with each type}
				\label{tbl:type-metadata}
			\end{table}
			
			\begin{figure}
				\begin{singlespaced}
					\begin{pygments}{clj}
(with-meta
  {:code    "cdps 101"
   :title   "Human-Mutant Relations"
   :subject "cdps"}
  {:type  :entity
   :class :course
   :id    "course|cdps_101"})
					\end{pygments}
				\end{singlespaced}
				
				\caption{Internal representation of named tuple from \cref{tbl:hmr-properties-rel}}
				\label{clj:with-meta-internal-rep}
			\end{figure}
			
			Once the internal representation is constructed, it may be transformed into a document.  The mapping of a map to document is trivial; a field is created for every key in the map and the value corresponding to the key is the value of the field.  Unfortunately Lucene does not facilitate the storage of metadata.  Therefore the system must deal with metadata in a different way.
			
			The system modifies each key of the metadata; two underscores are prepended and appended to the key name.  This allows the system to differentiate between metadata and attributes.  The transformation from internal representation to document is given in \cref{tbl:internal-rep-to-document}.
			
			\begin{table}
				\centering
				
				\begin{tabular}{ll}
					\toprule
					Field & Value \\
					\midrule
					code & cdps 101 \\
					title & human mutant relations \\
					subject & cdps \\
					\_\_type\_\_ & entity \\
					\_\_class\_\_ & course \\
					\_\_id\_\_ & course|cdps\_101 \\
					\bottomrule
				\end{tabular}
				
				\caption{Internal representation from \cref{clj:with-meta-internal-rep} in document form}
				\label{tbl:internal-rep-to-document}
			\end{table}
		
		\subsubsection{Indexing}
			With the named tuples transformed into documents, the index may be constructed.  The first step is to open the index for writing.  In Lucene, this is accomplished by creating an \texttt{IndexWriter} object on a \texttt{Directory} object that points to the index location.  The \texttt{IndexWriter} object also expects an analyzer to be used by default on documents it indexes.  The system uses a \texttt{WhitespaceAnalyzer} by default, but offers the ability to choose a different analyzer for specific fields.
			
			For every named tuple, the transformed document is written to the index by the index writer object.  In addition, the indexing document of every entity group is added.
		
	\subsection{Keyword Search in Document Space}
	\label{sec:keyword-search-document-space}
		With the entity graph encoded into the document model, users may begin issuing search queries.  Every query follows the following pattern:  users look up values (optionally using fuzzy search), these values are used to locate entities, and once two entities are selected, the system attempts to connect the two.
		
		\subsubsection{Approximate String Matching Using \(n\)-grams}
			Recall that entity values are stored as their \(n\)-gram (\cref{sec:n-gram}).  This allows users to make character substitutions, deletions, or additions, and still return values they may have intended on finding.  Without the use of \(n\)-grams, a misspelling on the user's part may result in an empty set of values being returned.  Values that are approximately matched would be assigned a lower score than those which are fully matched, but they would at least appear in the results.
			
			Rather than guessing the user's intention, the system presents the user with a list of values to give them the option of substituting their entry for an approximate match.  This autosuggest feature is intended to improve the user experience by eliminating a source of frustration -- irrelevant results as a result of a simple spelling error.  This list of values is created by applying \(n\)-gram analysis on the user query and searching the resulting stream of tokens against the collection of value documents.
		
		\subsubsection{Flexible Keyword Search \gls{api}}
			The system provides a simple -- yet flexible -- keyword search \gls{api}.  Recall the extended query (\cref{ex:extended-query}) is in the form
			\[
				\query(\field, \w)
			\]
			
			The search \gls{api} provides a function that accepts a field to search in, as well one or more words to search for.  A phrase query comprised of every word is constructed.
			\[
				\query(\field, \w_1, \w_2, \dotsc, \w_n) = \bigcap_{\w \in \{\w_1, \w_2, \dotsc, \w_n\}} \query(\field, \w)
			\]
			
			Another function, \texttt{boolean-query}, accepts one or more \(\query\) functions as well as a boolean operator for each and returns the result.  The symbols \texttt{:and}, \texttt{:or}, and \texttt{:not} provide \(\cap\), \(\cup\), and \(\neg\), respectively.
			
			\begin{ex}[\texttt{boolean-query}]
				For example, a query to find all documents with a subject of ``MATH'' (\cref{ex:extended-query}) would require two queries joined by \texttt{boolean-query}, as shown in \cref{clj:boolean-query}.
				
				\begin{figure}
					\begin{singlespaced}
						\begin{pygments}{clj}
(boolean-query
 [[:and (query :subject ``math'')]
  [:and (query :text ``class'')]])
						\end{pygments}
					\end{singlespaced}
					
					\caption{Boolean query in Clojure}
					\label{clj:boolean-query}
				\end{figure}
			\end{ex}

	\subsection{Graph Search in Document Space}
		With the ability for users to find relevant entities using fuzzy value search and the flexible keyword search \gls{api} (\cref{sec:keyword-search-document-space}), they must be able to find connections between two entities.  As previously stated, the document encoding of relational data is a graph.  This allows the system to search for links between entities by utilizing one or more graph search algorithms, such as \gls{bfs}.
		
		\subsubsection{A Case For Graph Search}
			Tuples are fragments, or facts, of larger pieces of knowledge.  By utilizing graph search, we amalgamate these facts to provide a user with a broader view.
			
				By utilizing graph search, we can take the facts in \cref{tbl:facts} and compose new facts.  For example, we could learn who teaches Complex Analysis.  It also allows us to ask questions, such as ``who taught Complex Analysis with Instructor X?''
			
			\begin{table}
				\begin{subtable}[b]{0.5\linewidth}
					\centering
					
					\begin{tabular}{ll}
						\toprule
						Field & Terms \\
						\midrule
						code & \(\{\text{math}, \text{360}\}\) \\
						title & \(\{\text{complex}, \text{analysis}\}\) \\
						subject & \(\{\text{math}\}\) \\
						\bottomrule
					\end{tabular}
					
					\caption{Fact representing a Course}
					\label{subtbl:fact-course}
				\end{subtable}
				\begin{subtable}[b]{0.5\linewidth}
					\centering
					
					\begin{tabular}{ll}
						\toprule
						Field & Terms \\
						\midrule
						id & \(\{\text{math}\}\) \\
						name & \(\{\text{mathematics}\}\) \\
						\bottomrule
					\end{tabular}
					
					\caption{Fact representing a Subject}
					\label{subtbl:fact-subject}
				\end{subtable}
				
				\caption{Fragments, or facts, of information in a dataset}
				\label{tbl:facts}
			\end{table}
			
			This automated discovery of relations between facts is why graph search is important.
		
		\subsubsection{Graph Search Algorithms}
			Recall we defined a graph as \(\sgraph{} = (\text{V}, \text{E})\), where \(\text{V}\) is the set of all facts in the database, and \(\text{E}(\egraph{})\) is the set of entities in entity group \(T\).  We say that two vertices in \(\egraph{}\) are connected if, and only if.
			\[
				\text{E}(\egraph{}) \cap \text{E}(\egraph{}') \not= \emptyset
			\]
			
			We must, given a keyword query \(\q\), find a subgraph, \(H\) of \(\sgraph{}\), such that
			
			\begin{itemize}
				\item The vertices, or hops, in \(H\) are minimized; and
				\item The satisfaction of keywords in \(\q\) by the vertices in \(H\) are maximized
			\end{itemize}
			
			To accomplish this, we find all vertices by entity group search and call this \(C\).  A graph search algorithm is used to minimally connect the vertices in \(C\).
			
			\begin{algorithm}[!ht]
				\caption{\(\textsc{Graph-Search}(C)\)}
				\label{alg:graph-search}
				
				\begin{singlespaced}
					\begin{algorithmic}[1]
						\REQUIRE \(C\) is a list of vertices
						\ENSURE minimal path between vertices in \(C\)
						\medskip
						\STATE \(s \in C\)
						
						\FOR{\(t \in C - \{s\}\)}
							\STATE find path connecting \(s \rightarrow t\) such that \(\textsc{Length}(\text{path}) \le \text{max-hops}\)
						\ENDFOR
						
						\RETURN path
						\medskip
						\medskip
					\end{algorithmic}
				\end{singlespaced}
			\end{algorithm}
		
		\subsubsection{Graph Search in Document Space}
			The question becomes:  why do we perform this graph search in document space?  Why not on the original relational space?  There are two main answers to this question:  speed and flexibility.
			
			Rather than relying on scanning every tuple in every relation in a relational database, the document model represents every tuple as a semi-structured document.  The contents of every field in these documents is indexed by an inverted list index data structure which permits fast lookup of documents based on keywords.  We utilize this property to quickly locate the initial ``source'' vertices.
			
			During the indexing process, analyzers are applied against the text.  These may, for instance, remove the suffix of words; a process called stemming \cite{porter-97}.
			
			\begin{ex}[Porter Stemmer]
				A course may have ``mathematics'' or even ``mathematical'' in the title.  By utilizing a stemmer, we match both.  The Porter stemming algorithm returns ``mathemat'' as the root word for both of the examples.
			\end{ex}
			
			Other analyzers may remove stop words.  Stop words may include ``and'', ``or'', and ``not''.  These are functional words that may be removed from the text corpus without adversely affecting the meaning whose presence may otherwise affect the quality of search results \cite{silva-03}.
			
			Our implementation uses a document to represent edges in the entity graph.  By using the document model, we are able to quickly locate all other vertices connected to a specific vertex.
			
			\begin{ex}[Search for Connected Entities]
				Given the indexing documents \(x_1, x_2, x_3\)
				
				\begin{align*}
					x_1[\text{``entities''}] &= \{\text{course|math\_360}, \text{subject|math}\} \\
					x_2[\text{``entities''}] &= \{\text{course|math\_101}, \text{subject|math}\} \\
					x_3[\text{``entities''}] &= \{\text{course|cdps\_101}, \text{subject|cdps}\} \\
				\end{align*}
				
				A query, \(\q\) for all entities related to the subject ``math''.
				\[
					\q = \query(\text{``entities'', ``subject|math''}) \cap \query(\text{``\_\_type\_\_'', ``group''})
				\]
				
				Yields the results \(x_1, x_2\).
			\end{ex}
			
			\begin{remark}
				The \_\_type\_\_ field indicates whether the document represents a value, entity, or entity group.
			\end{remark}
		
		\subsubsection{Breadth-First Search}
			\citeauthor*{cormen-09} define \gls{bfs} as follows
			
			\begin{displayquote}[\cite{cormen-09}]
				Given a graph \(G = (V, E)\) and a distinguished source vertex \(s\), breadth-first search systematically explores the edges of \(G\) to ``discover'' every vertex that is reachable from \(s\). It computes the distance (smallest number of edges) from \(s\) to each reachable vertex. It also produces a ``breadth-first tree'' with root \(s\) that contains all reachable vertices. For any vertex \(v\) reachable from \(s\), the path in the breadth-first tree from \(s\) to \(v\) corresponds to a ``shortest path'' from \(s\) to \(v\) in \(G\), that is, a path containing the smallest number of edges. The algorithm works on both directed and undirected graphs.
			\end{displayquote}
			
			\gls{bfs} populates the frontier before exploring the next hop.  As a result, \gls{bfs} is able to explore a large, sparsely connected graph quickly.  If the graph is dense, \gls{bfs} would consume a substantial amount of memory, making an algorithm such as \gls{dfs} more suitable.
			
		\subsubsection{Functional \gls{bfs}}
			The simple nature of \gls{bfs} makes it ideal to implement in a functional manner.  There is minimal shared state or side effects, allowing the search to be conducted recursively.  Newly discovered vertices are conjoined with the existing queue to form a new queue.  Due to Clojure's persistent data structures, this operation is more efficient than intuition would dictate.
			
			The functional implementation of \gls{bfs} combines state changes into larger units.  This leaves large segments of the implementation free of side effects.  We exploit this fact to implement \gls{bfs} concurrently in Clojure.
		
		\subsubsection{Concurrent \gls{bfs}}
			In our implementation, the exploration of adjacent nodes is performed concurrently.  Rather than exploring each node sequentially, as show in \cref{fig:concurrent-initial}, \cref{fig:concurrent-sequential}, and finally \cref{fig:concurrent-concurrent}, nodes ``course|math\_101'' and ``course|math\_360'' are explored simultaneously.
			
			\begin{figure}
				\centering
				\includegraphics[scale=0.9]{figures/graphs/concurrent/initial}
				
				\caption{Initial graph}
				\label{fig:concurrent-initial}
			\end{figure}
			
			\begin{figure}
				\centering
				\includegraphics[scale=0.9]{figures/graphs/concurrent/sequential}
				
				\caption{Exploring the first adjacent node sequentially}
				\label{fig:concurrent-sequential}
			\end{figure}
			
			\begin{figure}
				\centering
				\includegraphics[scale=0.9]{figures/graphs/concurrent/concurrent}
				
				\caption{Both adjacent nodes explored}
				\label{fig:concurrent-concurrent}
			\end{figure}
		
		%
		%Distributed functional description of BFS
		%
		%Implementation details:
		%- Ref
		%- Atoms
		%- Agent
		%
		%\begin{itemize}
		%	\item Search in document graph using graph search algorithms with functional implementations: (Ford Fulkerson, BFS)
		%	\item Speed up using concurrency
		%	\item Clojure specific optimization: ref + atom
		%\end{itemize}
	
	\chapter{Experimental Evaluation}
\label{chap:experimental-evaluation}
	In this chapter we evaluate our implementation of the system for transforming data in the relational model to the document model and vice versa described in \cref{chap:tale-of-two-data-models}.  We cover the implementation details in \cref{sec:implementation}, and the methodology and evaluation in \cref{sec:runtime-evaluation}.
	
	\section{Implementation}
	\label{sec:implementation}
		The system was implemented in Clojure, which \textcquote{clj-home}{is a dynamic programming language that targets the \gls{jvm}}.  Clojure was chosen due to its rich, immutable, and persistent data structures, excellent concurrency support, and seamless \gls{jvm} interoperability.  These features were discussed in detail in \cref{sec:features-of-clojure}.
		
		\subsection{Code Base Statistics}
			The system consists of over 800 lines of Clojure, along with approximately 550 lines of Python.  The Python code is used to construct the data set by crawling the course information site, as well as to aggregate the benchmark data produced by the system, producing graphs.
			
			All development has occurred on GitHub \cite{molly-repo}.  The use of Git and GitHub permits collaboration between researchers.  With the code publicly available, future researchers may study and run it.
	
	\section{The Data Corpus}
	\label{sec:data-corpus}
		The data corpus was derived from the \gls{uoit} mycampus database.  An \gls{html} crawler was written in Python that scraped the information from the \gls{uoit} class schedule search page.  This data was parsed, normalized, then placed in a SQLite database.
		
		The data corpus consists of numerous classes of objects.  These are:  courses (\cref{tbl:corpus-course}), instructors (\cref{tbl:corpus-instructor}), schedules (\cref{tbl:corpus-schedule}), sections (\cref{tbl:corpus-section}), and teaches (\cref{tbl:corpus-teaches}).  A graph representation of how these classes of objects are related can be found in \cref{fig:schema-graph}.  The data corpus is defined in \cref{chap:data-corpus-def}
		
		The number of objects, as of the publication of this thesis, can be found in \cref{tbl:data-corpus-count}.
		
		\begin{table}
			\centering
			\begin{tabular}{lr}
			\toprule
			Class & Count \\
			\midrule
			Course & 1340 \\
			Instructor & 849 \\
			Schedule & 25755 \\
			Section & 14463 \\
			Teaches & 15358 \\
			\bottomrule
			\end{tabular}
			
			\caption{Number of objects in data corpus, grouped by class}
			\label{tbl:data-corpus-count}
		\end{table}
	
	\section{Runtime Evaluation}
	\label{sec:runtime-evaluation}
		Scripts were written to coordinate the execution, collection, and transformation of the performance data of our implementation.
		
		\subsection{Methodology}
			We used Criterium\footnote{\url{http://hugoduncan.org/criterium/}} to handle the execution of the benchmarks as it handles unique concerns stemming from benchmarking on the \gls{jvm}.  These issues, identified by \citeauthor{rob-java-bench-08} \cite{rob-java-bench-08}, include:
			
			\begin{itemize}
				\item Statistical processing of multiple evaluations
				\item Inclusion of a warm-up period, designed to allow the JIT compiler to optimize its code
				\item Purging of the garbage collector before testing, to isolate timings from GC state prior to testing
				\item A final forced garbage collection after testing to estimate impact of cleanup on the timing results
			\end{itemize}
		
			This requires a much longer runtime as each function must be invoked numerous times.
			
			During evaluation, Criterium collects performance metrics.  Upon completion of the evaluation, it performs statistical analysis of these metrics using the bootstrap procedure developed by \citeauthor{efron-87} \cite{efron-87}.  These metrics include mean, samples, variance, quartiles, outliers, and more.
		
			\subsubsection{Data Collection}
			\label{sec:data-collection}
				The performance metrics computed by Criterium are returned as a Clojure map data structure.  The evaluation process may take several hours to complete, necessitating a separation between data collection and post-processing.  These metrics are stored offline for further processing.
				
				In order to utilize the Clojure output in Python, a data interchange format (\gls{json}) is used.  The benchmark function writes the Criterium performance analysis out as a \gls{json} string to stdout and the output is captured by the benchmark script.  An example of this \gls{json} output is given in \cref{fig:criterium-json-output}.
				
				\begin{figure}
					\centering % Pointless, but who knows in the future.
					\begin{verbatim}
                    [{
                        "max-hops": ...,
                        "method": ...,
                        "results": {
                            "execution-count": ...,
                            "final-gc-time": ...,
                            "lower-q": [...],
                            "mean": [...],
                        ...
                    }, ...]
					\end{verbatim}
					
					\caption{Partial \gls{json} output from Criterium.}
					\label{fig:criterium-json-output}
				\end{figure}
		
		\subsection{Performance}
		\label{sec:performance}
			Performance was measured for the various system components.  An analysis of the metrics collected is presented in this section.
			
			\subsubsection{Indexing}
				The indexing process is computationally intensive but short lived.  After the initial \gls{jvm} warmup period, the time required to construct the index scales with the number of named tuples and relations between them.
				
				\begin{table}
					\centering
					\begin{tabular}{ll}
						\toprule
						Number of Groups & Elapsed Time (s) \\
						\midrule
						0 & 11.800 \\
						1 & 12.446 \\
						2 & 19.771 \\
						\bottomrule
					\end{tabular}
					
					\caption{Indexing time growth by number of entity groups, averaged over 5 runs}
					\label{tbl:index-growth-entity-groups}
				\end{table}
				
				We see in \cref{tbl:index-growth-entity-groups} the indexing time increases minimally between 0 and 1 group.  The number of entity groups added by the first entity schema is relatively small.  Contrast this to the indexing time between 1 and 2 groups, which increases considerably.  The number of entity groups also grew considerably, explaining the time increase.
			
			\subsubsection{Graph Search}
				The worst-case performance of \gls{bfs} is \(\mathcal{O}(n^2)\).  This is reflected in \cref{fig:growth} which follows an exponential growth curve.  In an attempt to mitigate the rapid increase in search time, concurrent variants of \gls{bfs} were also implemented and benchmarked.
				
				\begin{figure}
					\centering
					\includegraphics[scale=0.9]{figures/charts/growth.pdf}
					\caption{Growth of each graph search algorithm implementation by number of hops}
					\label{fig:growth}
				\end{figure}
				
				We see in \cref{fig:growth} the rate of growth of \gls{bfs} is as expected.  The rate of growth of \gls{bfs} with references and \gls{bfs} with atoms is nearly linear.  The atom implementation is slightly more performant as it lacks some of the overhead associated with references.
				
				\begin{figure}
					\begin{subfigure}[b]{.5\linewidth}
						\includegraphics[scale=0.45]{figures/charts/1_hops.pdf}
						\caption{1 Hop}
						\label{subfig:1-hop}
					\end{subfigure}
					\begin{subfigure}[b]{.5\linewidth}
						\includegraphics[scale=0.45]{figures/charts/2_hops.pdf}
						\caption{2 Hops}
						\label{subfig:2-hops}
					\end{subfigure}
				\end{figure}
				\begin{figure}
					\ContinuedFloat
					\begin{subfigure}[b]{.5\linewidth}
						\includegraphics[scale=0.45]{figures/charts/3_hops.pdf}
						\caption{3 Hops}
						\label{subfig:3-hops}
					\end{subfigure}
					\begin{subfigure}[b]{.5\linewidth}
						\includegraphics[scale=0.45]{figures/charts/4_hops.pdf}
						\caption{4 Hops}
						\label{subfig:4-hops}
					\end{subfigure}
				\end{figure}
				
				\begin{figure}
					\ContinuedFloat
					\begin{subfigure}[b]{.5\linewidth}
						\includegraphics[scale=0.45]{figures/charts/5_hops.pdf}
						\caption{5 Hops}
						\label{subfig:5-hops}
					\end{subfigure}
					\begin{subfigure}[b]{.5\linewidth}
						\includegraphics[scale=0.45]{figures/charts/6_hops.pdf}
						\caption{6 Hops}
						\label{subfig:6-hops}
					\end{subfigure}
				\end{figure}
				\begin{figure}
					\ContinuedFloat
					\begin{subfigure}[b]{.5\linewidth}
						\includegraphics[scale=0.45]{figures/charts/7_hops.pdf}
						\caption{7 Hops}
						\label{subfig:7-hops}
					\end{subfigure}
					\begin{subfigure}[b]{.5\linewidth}
						\includegraphics[scale=0.45]{figures/charts/8_hops.pdf}
						\caption{8 Hops}
						\label{subfig:8-hops}
					\end{subfigure}
					
					\caption{Distribution of samples per method, broken down by hops}
					\label{fig:distribution-hops}
				\end{figure}
				
				\todo{All captions must be at least 10pt}
				
				The difference in rate of growth is further illustrated in \cref{fig:distribution-hops}.  As seen previously, \cref{subfig:1-hop} shows little difference in runtime between the three methods.  The difference becomes clearer in \cref{subfig:2-hops}, and by \cref{subfig:8-hops}, the difference is obvious.
				
				By using a profiling tool\footnote{\url{http://yourkit.com/}}, we see the behaviour of Clojure's concurrency implementation.
				
				\begin{figure}
					\centering
					
					\includegraphics{figures/images/threads}
					
					\caption{Thread count while running \gls{bfs} atom benchmark}
					\label{fig:runtime-threads}
				\end{figure}
				
				In \cref{fig:runtime-threads} we see that a number of threads are created and destroyed.  Recall a new thread is created every time the frontier is populated.
	
	\section{Web Interface}
	\label{sec:web-interface}
		In order to make the system more accessible to novice users, a web-based interface was created.
		
		The first step involves searching for entities by a value.  We use approximate (\(n\)-gram) string matching to find relevant values despite potential character substitutions, deletions, or additions.
		
		\begin{figure}
			\centering
			\includegraphics[scale=0.5]{figures/images/step-1}
			
			\caption{Approximate string matching of values}
			\label{fig:webui-step-1}
		\end{figure}
		
		Entities that match the desired values are displayed to the user.  They have the option of specifying the entity as either the source, or the target.
		
		\begin{figure}
			\centering
			\includegraphics[scale=0.5]{figures/images/step-2}
			
			\caption{Tabular display of entities}
			\label{fig:webui-step-2}
		\end{figure}
		
		When an entity is chosen, the navigation bar at the top of the page is updated to reflect the new selection.  This allows the user to hover their cursor over the respective element in order to remind themselves of their selection.
		
		\begin{figure}
			\centering
			\includegraphics[scale=0.5]{figures/images/step-3}
			
			\caption{Chosen entities are displayed at the top}
			\label{fig:webui-step-3}
		\end{figure}
		
		When both a source and target entity are selected, the user is able to search for the shortest path between them.  They are given the option of which graph search algorithm implementation to use.
		
		\begin{figure}
			\centering
			\includegraphics[scale=0.5]{figures/images/step-4}
			
			\caption{The algorithm implementation may be selected}
			\label{fig:webui-step-4}
		\end{figure}
		
		A short message displaying the search duration as well as memory consumption is followed by a series of tables representing the intermediate entities between the source and target entities.
		
		\begin{figure}
			\centering
			\includegraphics[scale=0.5]{figures/images/step-5}
			
			\caption{Result of a search between entities}
			\label{fig:webui-step-5}
		\end{figure}
		
		This interface allows users to query the database for information in a familiar manner.
		
	\section{Conclusion}
	\label{sec:eval-conclusion}
		In evaluating the system, we came to several conclusions.
		
		Benchmarking any code is difficult.  The process may not have exclusive control over the processor, memory is paged in and out, disk I/O is cached, etc.  The \gls{jvm} complicates matters with \gls{jit} compilation and garbage collection.
		
		The growth of \gls{bfs} can be mitigated by the use of concurrency.  Clojure facilitated a natural transition from a classical implementation of \gls{bfs} to a highly concurrent one.
	
	\chapter{Conclusion}
\label{chap:conclusion}
	\section{Summary}
		While the relational model is a powerful and well understood method of storing data, it is not without its shortcomings.  The rigidity of the relational model comes at the cost of usability.  A change to the data model may require a rewrite of queries to account for the different join paths, increasing the cognitive burden on users.
		
		In contrast, the document model represents semi-structured and unstructured data.  The queries issued against the document model are unstructured and flexible.  This allows users with little or no prior domain knowledge to issue queries.  Unfortunately this flexibility comes at the cost of foreign keys, data consistency, and aggregate queries.
		
		In \cref{chap:tale-of-two-data-models} we introduced the relational and document data models.  We compared and contrasted the two data models, paving the way for our contribution.  We introduced our contribution in \cref{chap:best-of-both-worlds}, providing a formal definition of a framework for the translation between the relational and document data models.  Our implementation was introduced in \cref{chap:along-came-clojure}, which also covered Clojure.  In \cref{chap:experimental-evaluation}, we evaluated our implementation, identifying performance characteristics.
		
	\section{Lessons Learned}
		The system evaluation in \cref{chap:experimental-evaluation} yielded several important insights.
		
		\begin{itemize}
			\item We have found that the simplest algorithms are the easiest to parallelize.  The reduced complexity, and thus state, reduces the amount of shared data that must be synchronized.  This allows for higher concurrency.
			\item Clojure's \gls{stm} implementation is simple to use and effective.  A few simple functions provide a powerful concurrency model.
			\item Sometimes a simpler approach to concurrency is the most appropriate one.  In our evaluation, atoms provided better performance than references.  Atoms allow for finer granularity in concurrency, reducing the overhead associated with references.  This is desirable in situations that do not require much shared state.
			\item Clojure is a powerful language that encouraged us to write correct code first, then optimize it later.  The transition to a concurrent implementation of \gls{bfs} was trivial.  The switch from atoms to references was also trivial.
		\end{itemize}

	\appendix
	
	\begin{singlespaced}
		% !TEX root = Thesis.tex
	
\chapter{Source Code}
	Each namespace in the code is divided into sections in the thesis document.
	  
	\section{molly}
		\subsection{molly.core}
			\inputpygments{clj}{../../src/clj/molly/core.clj}
	
	\clearpage
	\section{molly.conf}
		\subsection{molly.conf.config}
			\inputpygments{clj}{../../src/clj/molly/conf/config.clj}
		  
		\clearpage
		\subsection{molly.conf.mycampus}
			\inputpygments{clj}{../../src/clj/molly/conf/mycampus.clj}
	
	\clearpage
	\section{molly.datatypes}
		\subsection{molly.datatypes.database}
			\inputpygments{clj}{../../src/clj/molly/datatypes/database.clj}
		
		\clearpage
		\subsection{molly.datatypes.entity}
			\inputpygments{clj}{../../src/clj/molly/datatypes/entity.clj}
		
		\clearpage
		\subsection{molly.datatypes.schema}
			\inputpygments{clj}{../../src/clj/molly/datatypes/schema.clj}
	
	\clearpage
	\section{molly.index}
		\subsection{molly.index.build}
			\inputpygments{clj}{../../src/clj/molly/index/build.clj}
	
	\clearpage
	\section{molly.util}
		\subsection{molly.util.nlp}
			\inputpygments{clj}{../../src/clj/molly/util/nlp.clj}
	
	\clearpage
	\section{molly.search}
		\subsection{molly.search.lucene}
			\inputpygments{clj}{../../src/clj/molly/search/lucene.clj}
		
		\clearpage
		\subsection{molly.search.query\_builder}
			\inputpygments{clj}{../../src/clj/molly/search/query_builder.clj}
	
	\clearpage
	\section{molly.server}
		\subsection{molly.server.core}
			\inputpygments{clj}{../../src/clj/molly/server/core.clj}
		
		\clearpage
		\subsection{molly.server.remotes}
			\inputpygments{clj}{../../src/clj/molly/server/remotes.clj}
		
		\clearpage
		\subsection{molly.server.search}
			\inputpygments{clj}{../../src/clj/molly/server/search.clj}
		
		\clearpage
		\subsection{molly.server.util}
			\inputpygments{clj}{../../src/clj/molly/server/util.clj}
	
	\clearpage
	\section{molly.algo}
		\subsection{molly.algo.common}
			\inputpygments{clj}{../../src/clj/molly/algo/common.clj}
		
		\clearpage
		\subsection{molly.algo.bfs}
			\inputpygments{clj}{../../src/clj/molly/algo/bfs.clj}
		
		\clearpage
		\subsection{molly.algo.bfs\_atom}
			\inputpygments{clj}{../../src/clj/molly/algo/bfs_atom.clj}
		
		\clearpage
		\subsection{molly.algo.bfs\_ref}
			\inputpygments{clj}{../../src/clj/molly/algo/bfs_ref.clj}
	
	\clearpage
	\section{molly.bench}
		\subsection{molly.bench.benchmark}
			\inputpygments{clj}{../../src/clj/molly/bench/benchmark.clj}
	\end{singlespaced}
	
	\chapter{Data Corpus}
\label{chap:data-corpus-def}
	Tables representing various object classes of data corpus used in thesis implementation and evaluation.
	
	\begin{table}[H]
		\begin{subtable}[b]{.5\linewidth}
			\centering
			\begin{tabular}{ll}
				\toprule
				Property & Data Type \\
				\midrule
				id & INT \\
				name & VARCHAR \\
				\bottomrule
			\end{tabular}
			
			\caption{Campus Data Structure}
			\label{tbl:corpus-campus}
		\end{subtable}
		\begin{subtable}[b]{.5\linewidth}
			\centering
			\begin{tabular}{ll}
				\toprule
				Property & Data Type \\
				\midrule
				id & INT \\
				name & VARCHAR \\
				\bottomrule
			\end{tabular}
			
			\caption{Instructor Data Structure}
			\label{tbl:corpus-instructor}
		\end{subtable}
		\begin{subtable}[b]{.5\linewidth}
			\centering
			\begin{tabular}{ll}
				\toprule
				Property & Data Type \\
				\midrule
				code & VARCHAR \\
				title & VARCHAR \\
				subject\_id & VARCHAR \\
				\bottomrule
			\end{tabular}
			
			\caption{Course Data Structure}
			\label{tbl:corpus-course}
		\end{subtable}
		\begin{subtable}[b]{.5\linewidth}
			\centering
			\begin{tabular}{ll}
				\toprule
				Property & Data Type \\
				\midrule
				id & INT \\
				name & VARCHAR \\
				campus\_id & VARCHAR \\
				\bottomrule
			\end{tabular}
			
			\caption{Location Data Structure}
			\label{tbl:corpus-location}
		\end{subtable}
	\end{table}
	
	\begin{table}
		\ContinuedFloat
		\begin{subtable}[b]{.5\linewidth}
			\centering
			\begin{tabular}{ll}
				\toprule
				Property & Data Type \\
				\midrule
				id & VARCHAR \\
				name & VARCHAR \\
				\bottomrule
			\end{tabular}
			
			\caption{Subject Data Structure}
			\label{tbl:corpus-subject}
		\end{subtable}
		\begin{subtable}[b]{.5\linewidth}
			\centering
			\begin{tabular}{ll}
				\toprule
				Property & Data Type \\
				\midrule
				id & VARCHAR \\
				name & VARCHAR \\
				\bottomrule
			\end{tabular}
			
			\caption{Term Data Structure}
			\label{tbl:corpus-term}
		\end{subtable}
		\begin{subtable}[b]{.5\linewidth}
			\centering
			\begin{tabular}{ll}
				\toprule
				Property & Data Type \\
				\midrule
				id & INT \\
				days & VARCHAR \\
				sch\_type & VARCHAR \\
				date\_start & DATE \\
				date\_end & DATE \\
				week & VARCHAR \\
				time\_start & TIME \\
				time\_end & TIME \\
				location\_id & INT \\
				section\_id & INT \\
				instructor\_id & INT \\
				\bottomrule
			\end{tabular}
			
			\caption{Schedule Data Structure}
			\label{tbl:corpus-schedule}
		\end{subtable}
		\begin{subtable}[b]{.5\linewidth}
			\centering
			\begin{tabular}{ll}
				\toprule
				Property & Data Type \\
				\midrule
				crn & VARCHAR \\
				reg\_start & DATE \\
				reg\_end & DATE \\
				credits & FLOAT \\
				section\_num & VARCHAR \\
				term\_id & VARCHAR \\
				course\_code & VARCHAR \\
				levels & VARCHAR \\
				\bottomrule
			\end{tabular}
			
			\caption{Section Data Structure}
			\label{tbl:corpus-section}
		\end{subtable}
		
		\caption{Structure of data corpus}
		\label{tbl:data-corpus-defn}
	\end{table}
	
	\printbibliography
\end{document}