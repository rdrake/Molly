% !TEX root = Thesis.tex
\section{Relational Model}
	In its most basic form, the relational data model is built upon sets and tuples.  Each of these sets consist of a set of finite possible values.  Tuples are constructed from these sets to form relations.
	
	\begin{defn}[Named Tuple]
	\label{def:named-tuple}
		A named tuple \(\tuple\) is an instance of a relation \(\relation\), consisting of values corresponding to the attributes of \(\relation\).  For example,
	\end{defn}
	
	\begin{ex}
		Given a tuple \(\tuple = \{\text{code}: \text{``CDPS 101''}, \text{title}: \text{``Human-Mutant Relations''}, \text{subject}: \text{``CDPS''}\}\), we denote the attributes of \(\tuple\) as \(\attributes{\tuple} = \{\text{code}, \text{title}, \text{subject}\}\).  The values are \(\tuple\lbrack\text{code}\rbrack = \text{``CDPS 101''}\), \(\tuple\lbrack\text{title}\rbrack = \text{``Human-Mutant Relations''}\), and \(\tuple\lbrack\text{subject}\rbrack = \text{``CDPS''}\).
	\end{ex}
	
	\begin{defn}[Relation]
	\label{def:relation}
		A relation \(\relation\) is a set of named tuples, \(\relation = \{\tuple_1, \tuple_2, \dotsc, \tuple_n\}\), such that all the named tuples share the same attributes.
		\begin{equation}
			\forall \tuple, \tuple' \in \relation, \attributes{\tuple} = \attributes{\tuple'}
		\end{equation}
	\end{defn}
	
	\begin{ex}
		An example Course relation, \(\relation\), would be
		\[
			\relation = \left\{
				\begin{array}{llllll}
					\{\text{code} &: \text{``CDPS 101''}, &\text{title} &: \text{``Human-Mutant Relations''}, &\text{subject} &: \text{``CDPS''}\}, \\
					\{\text{code} &: \text{``CDPS 201''}, &\text{title} &: \text{``Humans and You''}, &\text{subject} &: \text{``CDPS''}\}, \\
					\{\text{code} &: \text{``MATH 360''}, &\text{title} &: \text{``Complex Analysis''}, &\text{subject} &: \text{``MATH''}\}
				\end{array}
			\right\}
		\]
	\end{ex}
		
	Relations are typically represented as tables.
	
	\begin{table}
		\centering
		
		\begin{tabular}{lll}
			\toprule
			code & title & subject \\
			\midrule
			CDPS 101 & Human-Mutant Relations & CDPS \\
			CDPS 201 & Humans and You & CDPS \\
			MATH 360 & Complex Analysis & MATH \\
			\bottomrule
		\end{tabular}
		
		\caption{Course relation}
		\label{tbl:course-relation}
	\end{table}
	
	\begin{defn}[Keys]
	\label{def:keys}
		Keys are constraints imposed on relations.	A key constraint \(\key\) on a relation \(\relation\) is a subset of \(\attributes{\relation}\) which may uniquely identify a tuple.	Formally, we say \(\relation\) satisfies the key constraint \(\key\), denoted as \(\relation \models \key\), subject to
		\[
			\forall \tuple, \tuple' \in \relation, \tuple \not= \tuple' \implies \tuple\lbrack\key\rbrack \not= \tuple'\lbrack\key\rbrack
		\]
		
		For example, in \vref{tbl:course-relation}, the relation satisfies the key constraint \(\{\text{code}\}\) or \(\{\text{title}\}\), but not \(\{\text{subject}\}\).
	\end{defn}
	
	\begin{defn}[Foreign Keys]
	\label{def:foreign-keys}
		A \gls{fk} constraint applies to two relations, \(\relation_1, \relation_2\).  It asserts that values of certain attributes of \(\relation_1\) must appear as values of some corresponding attributes of \(\relation_2\).  A \gls{fk} constraint is written as
		\[
			\theta = \relation_1(\attribute_{11}, \attribute_{12}, \dotsc, \attribute_{1k}) \rightarrow \relation_2(\attribute_{21}, \attribute_{22}, \dotsc, \attribute_{2k})
		\]
		
		where \(\attribute_{1i} \subseteq \attributes{\relation_1}\) and \(\attribute_{2i} \subseteq \attributes{\relation_2}\).  We say \((\relation_1, \relation_2)\) satisfies \(\theta\), denoted as \((\relation_1, \relation_2) \models \theta\), if for every tuple \(\tuple \in \relation_1\), there exists a tuple \(\tuple' \in \relation_2\) such that \(\tuple\lbrack\attribute_{11}, \attribute_{12}, \dotsc, \attribute_{1k}\rbrack = \tuple'\lbrack\attribute_{21}, \attribute_{22}, \dotsc, \attribute_{2k}\rbrack\).
		
		We say \(\relation_1\) is the source, while \(\relation_2\) is the target.
				
		\begin{ex}
			Suppose we have a relation \(\rel{Course}{code, title, subject}\).	 We impose a \gls{fk} constraint of
			\begin{equation}
				\theta = \rel{Course}{subject} \rightarrow \rel{Subject}{id}
			\end{equation}
			
			which asserts \((\text{Course}, \text{Subject}) \models \theta\).  Therefore, if
			\[
				\tuple = \{\text{code}: \text{``CDPS 101''}, \text{title}: \text{``Human-Mutant Relations''}, \text{subject}: \text{``CDPS''}\}
			\]
			
			then \(\exists! \tuple' \in \text{Subject}\) such that \(\tuple'\lbrack\text{id}\rbrack = \text{``CDPS''}\).
		\end{ex}
	\end{defn}
	
	\begin{defn}[Relational Database]
	\label{def:relational-database}
		A relational database, \(\db\), is a named collection of relations (as defined by \vref{def:relation}), keys (as defined by \vref{def:keys}), and foreign key constraints (as defined by \vref{def:foreign-keys}).
		
		We use \(\name{\db}\) to denote the name of \(\db\), \(\relations{\db}\) the list of relations in \(\db\), \(\keys{\db}\) the list of key constraints of \(\db\), and \(\fks{\db}\) the list of foreign key constraints of \(\db\).
	\end{defn}
	
	\subsection{Schema Group}
		\begin{defn}[Schema Graph]
		\label{def:schema-graph}
			If we view relations as vertices, and foreign key constraints as edges, a database \(\db\) can be viewed as a \emph{schema graph} \(\sgraph{}\), formally defined as
			\begin{align}
				\text{vertices} &: \text{V}(\sgraph) = \relations{\db} \\
				\text{edges} &: \text{E}(\sgraph) = \fks{\db}
			\end{align}
		\end{defn}
		
		\begin{ex}
			Given the schema in \vref{fig:schema}
			
			\begin{figure}
				\centering
				
				\(\rel{Course}{\underline{code}, title, subject}\) \\
				\(\rel{Section}{\underline{id}, actual, campus, capacity, credits, levels, registration\_start, registration\_end, semester, sec\_code, sec\_number, year, course}\) \\
				\(\rel{Schedule}{\underline{id}, date\_start, date\_end, day, schedtype, hour\_start, hour\_end, min\_start, min\_end, classtype, location, section\_id}\) \\
				\(\rel{Instructor}{\underline{id}, name}\) \\
				\(\rel{Teaches}{\underline{id}, schedule\_id, instructor\_id, position}\) \\
				
				\caption{Subset of mycampus dataset schema}
				\label{fig:schema}
			\end{figure}
			
			\todo{Split over multiple lines}
			
			and the \gls{fk} constraints in \vref{eqn:fk-constraints}
			
			\begin{figure}
				\begin{align*}
					\rel{Schedule}{section\_id} &\rightarrow \rel{Section}{id} \\
					\rel{Section}{course} &\rightarrow \rel{Course}{code} \\
					\rel{Teaches}{schedule\_id} &\rightarrow \rel{Schedule}{id} \\
					\rel{Teaches}{instructor\_id} &\rightarrow \rel{Instructor}{id} \\
				\end{align*}
				
				\caption{\gls{fk} constraints on schema in \vref{fig:schema}}
				\label{eqn:fk-constraints}
			\end{figure}
			
			we produce the schema graph in \vref{fig:schema-graph}
			
			\begin{figure}
				\centering
				
				\begin{dot2tex}[neato]
					digraph G {
						node [shape=plaintext]; Course; Section; Schedule; Instructor; Teaches;
						
						Schedule -> Section;
						Section -> Course;
						Teaches -> Schedule;
						Teaches -> Instructor;
					}
				\end{dot2tex}
				
				\caption{Graph representation of relations (\cref{fig:schema}) and \gls{fk} (\cref{eqn:fk-constraints})}
				\label{fig:schema-graph}
			\end{figure}
		\end{ex}
		
		The relational data model is particularly powerful for analytic queries.  Given the schema graph in \vref{fig:schema-graph}, one can formulate the following analytic queries in a query language known as \gls{sql}.
			
		\begin{ex}
			Using \gls{sql}, find all section CRNs for the subject titled ``Community Development \& Policy Studies.''
			
			\begin{figure}
				\begin{singlespaced}
					\begin{pygments}{sql}
SELECT section.crn
FROM   section 
       JOIN course 
         ON section.course_code = course.code 
       JOIN subject 
         ON subject.id = course.subject_id 
WHERE  subject.name = 'Community Development & Policy Studies'; 
					\end{pygments}
				\end{singlespaced}
				
				\caption{Query to find section CRNs for a subject name.}
				\label{fig:query-section-crns}
			\end{figure}
			
			\todo{Fix SQL to reference ``new'' schema}
			
			The \gls{sql} query in \vref{fig:query-section-crns} results in \vref{tbl:query-section-crns-output}.
			
			\begin{table}
				\centering
				
				\begin{tabular}{l}
					\toprule
					crn \\
					\midrule
					10000 \\
					10001 \\
					10002 \\
					\bottomrule
				\end{tabular}
				
				\caption{Results of the query in \vref{fig:query-section-crns}.}
				\label{tbl:query-section-crns-output}
			\end{table}
		\end{ex}
		
	\subsection{Entity Group}
		\begin{defn}[Entity Group]
		\label{def:entity-group}
			An entity group is a forest, \(\egraph\), of tuples interconnected by join conditions defined by the \gls{fk} constraints in the schema graph \(\sgraph\).
			
			Given two vertices \(\tuple, \tuple' \in \text{V}(\egraph)\), \(\exists \relation_1, \relation_2 \in \relations{\db}\) such that \(\tuple \in \relation_1\), \(\tuple'\in \relation_2\), and \((\relation_1, \relation_2)\in \sgraph\).  That is, \(\tuple\) and \(\tuple'\) belong to two relations that are connected by the schema graph.

			Let \(\relation_1(\attribute_{11}, \attribute_{12}, \dotsc, \attribute_{1k}) \to \relation_2(\attribute_{21}, \attribute_{22}, \dotsc, \attribute_{2k})\) be the \gls{fk} that connects \(\relation_1, \relation_2\).  We further assert that \(\tuple\lbrack\attribute_{11}, \attribute_{12}, \dotsc, \attribute_{1k}\rbrack = \tuple'\lbrack\attribute_{21}, \attribute_{22}, \dotsc, \attribute_{2k}\rbrack\).
		\end{defn}
		
		Entity groups define complex, structured objects that include more information than individual tuples in the relations.
		
		\begin{ex}
			The information in \vref{tbl:hmr-properties} all relates to the Course titled Human-Mutant Relations, however no single tuple in the database has all of this information as a result of database normalization.
			
			\begin{table}
				\centering
				
				\begin{tabular}{ll}
					\toprule
					Attribute & Value \\
					\midrule
					code & CDPS 101 \\
					title & Human-Mutant Relations \\
					subject & Community Development \& Policy Studies \\
					\bottomrule
				\end{tabular}
				
				\caption{Properties of the Course titled Human-Mutant Relations.}
				\label{tbl:hmr-properties}
			\end{table}
			
			We require an entity group (\vref{fig:hmr-entity-group}) to join together all pieces of information related to this course. 
			
			\begin{figure}
				\centering
				
				\begin{dot2tex}[dot]
digraph G {
	node [shape=plaintext]; "Human-Mutant Relations"; "Community Development & Policy Studies";
	
	"Human-Mutant Relations" -> "Community Development & Policy Studies";
}
				\end{dot2tex}
				
				\caption{Human-Mutant Relations entity group}
				\label{fig:hmr-entity-group}
			\end{figure}
		\end{ex}

	\subsection{Pros and Cons of the Relational Model}
		In order to better understand the motivation behind this work, it is important to examine both the strong and weak points of the relational model.
		
		\subsubsection{Pros}
			The enforcement of constraints is essential to the relational model.  There are several types of constraints, including uniqueness and \glspl{fk}.  The first constraint maintains uniqueness.
			
			The Course relation (\vref{tbl:course-relation}) has the attribute \texttt{code} as its primary key.	In order for other relations to reference a specific named tuple, the \texttt{code} attribute must be unique.
			
			\begin{ex}[Unique Constraint]
			\label{ex:unique-constraint}
				Attempt to insert another course with a \texttt{code} of ``CDPS 101.''
				
				\begin{singlespaced}
					\begin{pygments}{sql}
INSERT INTO course
VALUES      ('CDPS 101',
             'Mutant-Human Relations',
             'CDPS');
					\end{pygments}
				\end{singlespaced}
				
				The \gls{rdbms} enforces the primary key constraint on the \texttt{code} attribute, rejecting the insertion.
				
				\begin{verbatim}
Error: column code is not unique
				\end{verbatim}
			\end{ex}
			
			With the uniqueness of named tuples guaranteed (as demonstrated in \vref{ex:unique-constraint}), we must ensure that any named tuples that are referenced actually exist.  If they do not, the database must not permit the operation to continue.  Doing so would lead to dangling references.
			
			\begin{ex}[Referential Integrity]
				Attempt to insert the tuple (``CHEM 101'', ``Introductory Chemistry'', ``'CHEM'') in the Course relation.
				
				\begin{singlespaced}
					\begin{pygments}{sql}
INSERT INTO course
VALUES      ('CHEM 101',
             'Introductory Chemistry',
             'CHEM');
					\end{pygments}
				\end{singlespaced}
				
				Again we see the \gls{rdbms} protecting the integrity of the data.
				
				\begin{verbatim}
Error: foreign key constraint failed
				\end{verbatim}
			\end{ex}
			
			In addition to enforcing consistency, the relational model is capable of providing higher-level views of the data through aggregation.
			
			\begin{ex}[Aggregation]
				Find the number of sections offered for the subject named ``Community Development \& Policy Studies.''
				
				\begin{singlespaced}
					\begin{pygments}{sql}
SELECT Count(*)
FROM   section
       JOIN course
         ON section.course = course.code
       JOIN subject
         ON subject.id = course.subject
WHERE  subject.name = 'Community Development & Policy Studies';
					\end{pygments}
				\end{singlespaced}
			\end{ex}
			
			Information stored within a properly designed database is normalized.  That is, no information is repeated.
			
			\begin{ex}[Normalization]
				For example, suppose Emma Frost became headmistress and the subject named ``Community Development \& Policy Studies'' was renamed to ``Community Destruction \& Policy Studies.''  If this information were not normalized, each course in this subject would need to be updated.  Since this information is normalized, the following query will suffice.
				
				\begin{singlespaced}
					\begin{pygments}{sql}
UPDATE subject
SET    name = 'Community Destruction & Policy Studies'
WHERE  id = 'CDPS';
					\end{pygments}
				\end{singlespaced}
			\end{ex}
			
			The above examples are some of the most important reasons for choosing the relational model over others.	Unfortunately, the relational model is not without its downsides.
		
		\subsubsection{Cons}
			While the relational model excels at ensuring data consistency, aggregation, and reporting; it is not suitable for every task.	In order to issue queries, a user must be familiar with the schema.	 This requires specific domain knowledge of the data.
			
			An example of a complicated query involving two joins is give in \vref{fig:query-section-crns}.
			
			A casual user is unlikely to determine the correct join path, name of the tables, name of the attributes, etc.	This is in contrast to the document model, where the data is semi-structured or unstructured, requiring minimal domain knowledge.

			The relational model is also rigid in structure.  If a relation is modified, every query referencing said relation may require a rewrite.  Even a simple attribute being renamed (e.g.~\(\rho_{\text{name/alias}}(\text{Person})\)) is capable of modifying the join paths.  This rigidity places additional cognitive burden on users.
			
			In addition to having a rigid structure, most relational database management systems lack flexible string matching options.	 Assuming basic SQL-92 compliance, a \gls{rdbms} only supports the \texttt{LIKE} predicate \cite{sql-11}.
			
			\begin{ex}[\texttt{LIKE} Predicate]
				Find all courses with a title that contains ``man.''
				
				\begin{singlespaced}
					\begin{pygments}{sql}
SELECT *
FROM   course
WHERE  title LIKE '%man%';
					\end{pygments}
				\end{singlespaced}
			\end{ex}
			
			There are a couple of limitations to the \texttt{LIKE} predicate.  First, it only supports basic substring matching.  If a user accidentally searches for all courses with a title containing ``men,'' nothing would be found.
			
			Second, unless the predicate is applied to the end of the string and the column is indexed, performance will be poor.  The database must scan the entire table in order to answer the query, resulting in performance of \(\mathcal{O}(n)\), where \(n\) is the number of named tuples in the relation.