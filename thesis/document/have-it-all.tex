\chapter{Best of Both Worlds}
    \section{Encoding Named Tuples into Documents}
    \label{sec:named-tuples-documents}
        Recall in the extended document model (\Vref{sec:extending-the-document-model}), a document $\doc$ consists of fields $\field_1, \field_2, \dotsc, \field_n$.  Using the extended document model, we are left with a straight forward mapping of a tuple $\tuple$ to document $\doc$.
        
        For tuple $\tuple$, every attribute $\attribute \in \attributes{\tuple}$ maps to field $\field$ in $\doc$.  Every attribute value must be analyzed into an indexable form in order to store it in a field.
        
        \begin{align}
            \attributes{\tuple} &\xrightarrow{analyzed} \fields{\doc} \\
            \attribute_1, \attribute_2, \dotsc, \attribute_n &\xrightarrow{analyzed} \field_1, \field_2, \dotsc, \field_n
        \end{align}
        
        We denote the document encoding of $\tuple$ as $\docs{\tuple}$.
    
    \section{Mapping of Entity Groups to Documents}
        Recall that an entity group (\Vref{def:entity-group}) is a forest $\egraph$ of tuples $t$ such that
        
        \[
            \forall \bigl(\tuple, \tuple'\bigr) \in \egraph, \tuple \nequal \tuple' \Rightarrow \relations{\tuple} \nequal \relations{\tuple'}
        \]
        
        That is, all tuples are from distinct relations.
        
        Given the restriction
        
        \[
            \forall \bigl(\relation, \relation'\bigr) \in \sgraph{\db}, \exists! \bigl(\relation, \relation'\bigr) \models \theta
        \]
        
        we assert that if $\bigl(\tuple, \tuple'\bigr) \in \egraph$, then $\bigl(\tuple, \tuple'\bigr) \in \relations{\tuple} \bowtie \relations{\tuple'}$.
        
        Let $V\bigl(\egraph\bigr)$ be the vertices of $\egraph$, $E\bigl(\egraph\bigr)$ be the edges of $\egraph$.
        
        \begin{claim}
        \label{clm:lossless}
            $\egraph$ can always be reconstructed from $V\bigl(\egraph\bigr)$ without loss of information.
        \end{claim}
        
        \begin{proof}
            Given $V\bigl(\egraph\bigr)$, we must reconstruct $E\bigl(\egraph\bigr)$ in order to complete $\egraph$.
            
            Choose any $\bigl(\tuple, \tuple'\bigr) \in V\bigl(\egraph\bigr)$.  If $\bigl(\relations{\tuple}, \relations{\tuple'}\bigr) \in \sgraph{\db}$, then $\bigl(\tuple, \tuple'\bigr)$ is an edge in $\egraph$.
            
            Recall our earlier assertion that $\sgraph{\db}$ is cycle-free and foreign keys must be unique.
        \end{proof}
        
    \section{Encoding an Entity Group as a Document Group}
        Given a entity group $\egraph$, we construct two or more documents in order to represent the entity group in the document model.
        
        For every $\tuple \in V\bigl(\egraph\bigr)$, we construct a document $\docs{\tuple}$ (\Vref{sec:named-tuples-documents}).  With each tuple $\tuple$ stored in the document collection $\dc$, we construct an additional document which stores the association information.
        
        Let $x$ be the indexing document of $\egraph$.
        
        \[
            x\bigl[\mathrm{``entities''}\bigr] = \bigcup_{\tuple \in V\left(\egraph\right)} \uid{\tuple}
        \]
        
        Thus, the encoding of $\egraph$ is defined as
        
        \[
            \egraph \xrightarrow{\mathrm{encode}} \Bigl\{\docs{\tuple} : \tuple \in V\bigl(\egraph\bigr)\Bigr\} \cup \Bigl\{x\Bigr\}
        \]
        
        It's easy to see that from encode(G) we can recover V(G), the tuples in G.
        
        By \Vref{clm:lossless}, this is sufficient to recover G entirely.
    
    \section{Encoding Attribute Values into Searchable Documents}
		Each value for user selected attributes are converted into $n$-grams, and stored in special documents.
	
	\section{Iterative Search Using Document Encodings}
		A document database supports fast and flexible keyword search queries.  A search query is characterized by $q = \bigl(f, w\bigr)$, where $f$ is a field name, and $w$ is a search phrase.
		
		Search[q] is the set of documents returned by the text index.  The search function allows us to do many things:
		
		\begin{enumerate}
			\item Suggest values, correcting spelling errors
			\item Given attribute values, search all relevant entities
			\item Search for all relevant entity groups of one or more entities, using the indexing document
			\item We can connect entities via entity groups using (hyper)graph search algorithms.
		\end{enumerate}