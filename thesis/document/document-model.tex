% !TEX root = Thesis.tex
\section{Document Model}
	In this section we formally define the document model.
	
	Documents are a unit of information.  The definition of unit can vary.  It may represent an email, a book chapter, a memo, etc.  Contained within each document is a set of terms.
	
	In contrast to the relational model, the document model represents semi-structured as well as unstructured data.  Examples of information suitable to the document model includes emails, memos, book chapters, etc.
	
	These pieces, or units, of information are broken into documents.  Groups of related documents (for example, a library catalogue) are referred to as a document collection.

	\begin{defn}[Terms and Document]
	\label{def:document}
		A term, $t$, is an indivisible string (e.g.~a proper noun, word, or a phrase).  A document, $d$, is a bag of words.  Let $\freq_{t, d}$ be the frequency of terms $t$ in document $d$.
		
		Let $T$\nomenclature{$T$}{all terms in document collection} denote all possible terms, and $\fcn{Bag}{T}$ be all possible bag of terms.
	\end{defn}
	
	\begin{remark}
		We use the bag-of-words model for documents.  This means that position information of terms in a document is irrelevant, but the frequency of terms are kept in the document.  Documents are non-distinct sets.
	\end{remark}
	
	\begin{defn}[Document Collection]
	\label{def:document-collection}
		A document collection $D$ is a set of documents, written $D = \left\{d_1, d_2, \dotsc, d_k\right\}$.  The size of $D$ is denoted $N$\nomenclature{$N$}{Number of documents in collection}.  The number of unique terms, or size of $T$, in $D$, is denoted $M$\nomenclature{$M$}{Size of $T$}.
	\end{defn}
	
	\begin{ex}
	\label{ex:superhero-documents}
		Consider the following short sentences.
		
		\begin{enumerate}
			\item Superman is strong on Earth and lives on Earth.
			\item Batman was born on Earth.
			\item Superwoman is fast on Earth.
			\item Superman was born on Krypton.
		\end{enumerate}
		
		Each sentence represents a document, giving us the following documents.
		
		\begin{eqnarray*}
			d_1 &=& \left\{\textrm{``and''}: 1, \textrm{``on''}: 2, \textrm{``is''}: 1, \textrm{``lives''}: 1, \textrm{``earth''}: 2, \textrm{``strong''}: 1, \textrm{``superman''}: 1\right\} \\
			d_2 &=& \left\{\textrm{``batman''}: 1, \textrm{``on''}: 1, \textrm{``was''}: 1, \textrm{``earth''}: 1, \textrm{``born''}: 1\right\} \\
			d_3 &=& \left\{\textrm{``on''}: 1, \textrm{``is''}: 1, \textrm{``superwoman''}: 1, \textrm{``fast''}: 1, \textrm{``earth''}: 1\right\} \\
			d_4 &=& \left\{\textrm{``krypton''}: 1, \textrm{``born''}: 1, \textrm{``on''}: 1, \textrm{``was''}: 1, \textrm{``superman''}: 1\right\} \\
		\end{eqnarray*}
	\end{ex}
	
	\subsection{Vectorization of Documents}
		One of the most fundamental approach for search documents is to treat documents as high dimensional vectors, and the document collection as a subset in a vector space.  The search query becomes a nearest neighbour query in a vector space equipped with a distance measure.
		
		The first step is to convert bag of terms into vectors.  The standard technique (cite:text book) uses a scoring function that measures the relative importance terms in documents.
		
		\begin{defn}[TF-IDF Score]
			The term frequency is the number of times a term $t$ appears in a document $d$, as given by $\freq_{t, d}$.  The document frequency of a term $t$, denoted by $\df_t$, is the number of documents in $D$ that contains $t$.  It is defined as
			
			$$\df_t = \mid \left\{d \in D: t \in d\right\} \mid$$
			
			The combined TF-IDF score of $t$ in a document $d$ is given by
			
			$$\tfidf_{D, t, d} = \frac{\freq_{t, d}}{\mid d \mid} \cdot \log{\frac{N}{\df_t}}$$
		\end{defn}
		
		\begin{remark}
			The first component, $\frac{\freq_{t, d}}{\mid d \mid}$, measures the importance of a term within a document.  It is normalized to account for document length.  The second component, $\log{\frac{N}{\df_t}}$, is a measure of the rarity of the term within the document collection $D$.
		\end{remark}
		
		\begin{ex}
			Using the documents from Example~\ref{ex:superhero-documents}, the TF-IDF scores are as follows.
			
			$$\bordermatrix{
				~ & d_1 & d_2 & d_3 & d_4 \cr
				\textrm{``and''} & 0.2857 & 0.0000 & 0.0000 & 0.0000 \cr
				\textrm{``on''} & 0.0000 & 0.0000 & 0.0000 & 0.0000 \cr
				\textrm{``superwoman''} & 0.0000 & 0.0000 & 0.4000 & 0.0000 \cr
				\textrm{``batman''} & 0.0000 & 0.4000 & 0.0000 & 0.0000 \cr
				\textrm{``is''} & 0.1429 & 0.0000 & 0.2000 & 0.0000 \cr
				\textrm{``fast''} & 0.0000 & 0.0000 & 0.4000 & 0.0000 \cr
				\textrm{``born''} & 0.0000 & 0.2000 & 0.0000 & 0.2000 \cr
				\textrm{``krypton''} & 0.0000 & 0.0000 & 0.0000 & 0.4000 \cr
				\textrm{``earth''} & 0.1186 & 0.0830 & 0.0830 & 0.0000 \cr
				\textrm{``lives''} & 0.2857 & 0.0000 & 0.0000 & 0.0000 \cr
				\textrm{``strong''} & 0.2857 & 0.0000 & 0.0000 & 0.0000 \cr
				\textrm{``was''} & 0.0000 & 0.2000 & 0.0000 & 0.2000 \cr
				\textrm{``superman''} & 0.1429 & 0.0000 & 0.0000 & 0.2000 \cr
			}$$
		\end{ex}

		\begin{defn}
			Given a document collection $D$ with $M$ unique terms $T = \left[ t_1, t_2, \dotsc, t_n \right]$, each document $d$ can be represented by an $M$-dimensional vector.
			
			$$
				\vec{d} = 
				\left[
				\begin{array}{c}
					\tfidf_{t_1, d} \\
					\tfidf_{t_2, d} \\
					\vdots \\
					\tfidf_{t_n, d}
				\end{array}
				\right]
			$$
		\end{defn}
		
		\begin{ex}
			The documents in Example~\ref{ex:superhero-documents} would produce the following vectors.
			
			$$
			\vec{d_1} = 
				\left[
					\begin{array}{c}
						0.2857 \\
						0.0000 \\
						0.0000 \\
						0.0000 \\
						0.1429 \\
						0.0000 \\
						0.0000 \\
						0.0000 \\
						0.1186 \\
						0.2857 \\
						0.2857 \\
						0.0000 \\
						0.1429 \\
					\end{array}
				\right],
			\vec{d_2} = 
				\left[
					\begin{array}{c}
						0.0000 \\
						0.0000 \\
						0.0000 \\
						0.4000 \\
						0.0000 \\
						0.0000 \\
						0.2000 \\
						0.0000 \\
						0.0830 \\
						0.0000 \\
						0.0000 \\
						0.2000 \\
						0.0000 \\
					\end{array}
				\right],
			\vec{d_3} = 
				\left[
					\begin{array}{c}
						0.0000 \\
						0.0000 \\
						0.4000 \\
						0.0000 \\
						0.2000 \\
						0.4000 \\
						0.0000 \\
						0.0000 \\
						0.0830 \\
						0.0000 \\
						0.0000 \\
						0.0000 \\
						0.0000 \\
					\end{array}
				\right],
			\vec{d_4} = 
				\left[
					\begin{array}{c}
						0.0000 \\
						0.0000 \\
						0.0000 \\
						0.0000 \\
						0.0000 \\
						0.0000 \\
						0.2000 \\
						0.4000 \\
						0.0000 \\
						0.0000 \\
						0.0000 \\
						0.2000 \\
						0.2000 \\
					\end{array}
				\right]
			$$
		\end{ex}

	
	\subsection{Extending the Document Model}

\section{Pro and con of document model (1 day)}
	\begin{itemize}
		\item Good: exploratory queries using keywords (google)
		\item Good: easy (or no) syntax
		\item Good: fuzzy matching (using n-gram)
		\item Bad: No analytics
	\end{itemize}
		
		\section{Best of both worlds (4 days, week 3)}
	\begin{itemize}
		\item Hybrid database defined by both the relational model and the document model
		\item Translation between relational objects (entities and entity) groups to documents.
		\item Translation of documents back to relational objects.
		\item Proof of lossless translation between relational space and document space
	\end{itemize}