% !TEX root = Thesis.tex
\chapter{A Tale of Two Data Models}
The term ``data model'' refers to a notation for describing data and/or information.  It consists of the data structure, operations that may be performed on the data, as well as constraints placed on the data \cite{dbsys-06}.

In this chapter we provide a formal definition of the relational data model, discuss its merits, its shortcomings, and contrast it to the document data model.  Contrary to the relational model, the document model permits fast and flexible keyword search without requiring explicit domain knowledge of the data.  In addition, we demonstrate the feasibility of encoding a relational model into a document model in a lossless manner.

% !TEX root = Thesis.tex
\section{Relational Model}
	In its most basic form, the relational data model is built upon sets and tuples.  Each of these sets consist of a set of finite possible values.  Tuples are constructed from these sets to form relations.
	
	\begin{defn}[Named Tuple]
	\label{def:named-tuple}
		A named tuple \(\tuple\) is an instance of a relation \(\relation\), consisting of values corresponding to the attributes of \(\relation\).  For example,
	\end{defn}
	
	\begin{ex}
		Given a tuple \(\tuple = \{\text{code}: \text{``CDPS 101''}, \text{title}: \text{``Human-Mutant Relations''}, \text{subject}: \text{``CDPS''}\}\), we denote the attributes of \(\tuple\) as \(\attributes{\tuple} = \{\text{code}, \text{title}, \text{subject}\}\).  The values are \(\tuple\lbrack\text{code}\rbrack = \text{``CDPS 101''}\), \(\tuple\lbrack\text{title}\rbrack = \text{``Human-Mutant Relations''}\), and \(\tuple\lbrack\text{subject}\rbrack = \text{``CDPS''}\).
	\end{ex}
	
	\begin{defn}[Relation]
	\label{def:relation}
		A relation \(\relation\) is a set of named tuples, \(\relation = \{\tuple_1, \tuple_2, \dotsc, \tuple_n\}\), such that all the named tuples share the same attributes.
		\begin{equation}
			\forall \tuple, \tuple' \in \relation, \attributes{\tuple} = \attributes{\tuple'}
		\end{equation}
	\end{defn}
	
	\begin{ex}
		An example Course relation, \(\relation\), would be
		\[
			\relation = \left\{
				\begin{array}{llllll}
					\{\text{code} &: \text{``CDPS 101''}, &\text{title} &: \text{``Human-Mutant Relations''}, &\text{subject} &: \text{``CDPS''}\}, \\
					\{\text{code} &: \text{``CDPS 201''}, &\text{title} &: \text{``Humans and You''}, &\text{subject} &: \text{``CDPS''}\}, \\
					\{\text{code} &: \text{``MATH 360''}, &\text{title} &: \text{``Complex Analysis''}, &\text{subject} &: \text{``MATH''}\}
				\end{array}
			\right\}
		\]
	\end{ex}
		
	Relations are typically represented as tables.
	
	\begin{table}
		\centering
		
		\begin{tabular}{lll}
			\toprule
			code & title & subject \\
			\midrule
			CDPS 101 & Human-Mutant Relations & CDPS \\
			CDPS 201 & Humans and You & CDPS \\
			MATH 360 & Complex Analysis & MATH \\
			\bottomrule
		\end{tabular}
		
		\caption{Course relation}
		\label{tbl:course-relation}
	\end{table}
	
	\begin{defn}[Keys]
	\label{def:keys}
		Keys are constraints imposed on relations.	A key constraint \(\key\) on a relation \(\relation\) is a subset of \(\attributes{\relation}\) which may uniquely identify a tuple.	Formally, we say \(\relation\) satisfies the key constraint \(\key\), denoted as \(\relation \models \key\), subject to
		\[
			\forall \tuple, \tuple' \in \relation, \tuple \not= \tuple' \implies \tuple\lbrack\key\rbrack \not= \tuple'\lbrack\key\rbrack
		\]
		
		For example, in \vref{tbl:course-relation}, the relation satisfies the key constraint \(\{\text{code}\}\) or \(\{\text{title}\}\), but not \(\{\text{subject}\}\).
	\end{defn}
	
	\begin{defn}[Foreign Keys]
	\label{def:foreign-keys}
		A \gls{fk} constraint applies to two relations, \(\relation_1, \relation_2\).  It asserts that values of certain attributes of \(\relation_1\) must appear as values of some corresponding attributes of \(\relation_2\).  A \gls{fk} constraint is written as
		\[
			\theta = \relation_1(\attribute_{11}, \attribute_{12}, \dotsc, \attribute_{1k}) \rightarrow \relation_2(\attribute_{21}, \attribute_{22}, \dotsc, \attribute_{2k})
		\]
		
		where \(\attribute_{1i} \subseteq \attributes{\relation_1}\) and \(\attribute_{2i} \subseteq \attributes{\relation_2}\).  We say \((\relation_1, \relation_2)\) satisfies \(\theta\), denoted as \((\relation_1, \relation_2) \models \theta\), if for every tuple \(\tuple \in \relation_1\), there exists a tuple \(\tuple' \in \relation_2\) such that \(\tuple\lbrack\attribute_{11}, \attribute_{12}, \dotsc, \attribute_{1k}\rbrack = \tuple'\lbrack\attribute_{21}, \attribute_{22}, \dotsc, \attribute_{2k}\rbrack\).
		
		We say \(\relation_1\) is the source, while \(\relation_2\) is the target.
				
		\begin{ex}
			Suppose we have a relation \(\rel{Course}{code, title, subject}\).	 We impose a \gls{fk} constraint of
			\begin{equation}
				\theta = \rel{Course}{subject} \rightarrow \rel{Subject}{id}
			\end{equation}
			
			which asserts \((\text{Course}, \text{Subject}) \models \theta\).  Therefore, if
			\[
				\tuple = \{\text{code}: \text{``CDPS 101''}, \text{title}: \text{``Human-Mutant Relations''}, \text{subject}: \text{``CDPS''}\}
			\]
			
			then \(\exists! \tuple' \in \text{Subject}\) such that \(\tuple'\lbrack\text{id}\rbrack = \text{``CDPS''}\).
		\end{ex}
	\end{defn}
	
	\begin{defn}[Relational Database]
	\label{def:relational-database}
		A relational database, \(\db\), is a named collection of relations (as defined by \vref{def:relation}), keys (as defined by \vref{def:keys}), and foreign key constraints (as defined by \vref{def:foreign-keys}).
		
		We use \(\name{\db}\) to denote the name of \(\db\), \(\relations{\db}\) the list of relations in \(\db\), \(\keys{\db}\) the list of key constraints of \(\db\), and \(\fks{\db}\) the list of foreign key constraints of \(\db\).
	\end{defn}
	
	\subsection{Schema Group}
		\begin{defn}[Schema Graph]
		\label{def:schema-graph}
			If we view relations as vertices, and foreign key constraints as edges, a database \(\db\) can be viewed as a \emph{schema graph} \(\sgraph{}\), formally defined as
			\begin{align}
				\text{vertices} &: \text{V}(\sgraph) = \relations{\db} \\
				\text{edges} &: \text{E}(\sgraph) = \fks{\db}
			\end{align}
		\end{defn}
		
		\begin{ex}
			Given the schema in \vref{fig:schema}
			
			\begin{figure}
				\centering
				
				\(\rel{Subject}{\underline{id}, name}\) \\
				\(\rel{Course}{\underline{code}, title, subject}\) \\
				\(\rel{Term}{\underline{id}, name}\) \\
				\(\rel{Section}{\underline{crn}, term, course}\) \\
				\(\rel{Schedule}{\underline{id}, days, sch\_type, time\_start, time\_end, location, section, instructor}\) \\
				\(\rel{Instructor}{\underline{id}, name}\) \\
				
				\caption{Subset of mycampus dataset schema}
				\label{fig:schema}
			\end{figure}
			
			and the \gls{fk} constraints in \vref{eqn:fk-constraints}
			
			\begin{figure}
				\begin{align*}
					\rel{Course}{subject} &\rightarrow \rel{Subject}{id} \\
					\rel{Section}{term} &\rightarrow \rel{Term}{id} \\
					\rel{Section}{course} &\rightarrow \rel{Course}{code} \\
					\rel{Schedule}{section} &\rightarrow \rel{Section}{crn} \\
					\rel{Schedule}{instructor} &\rightarrow \rel{Instructor}{id}
				\end{align*}
				
				\caption{\gls{fk} constraints on schema in \vref{fig:schema}}
				\label{eqn:fk-constraints}
			\end{figure}
			
			we produce the schema graph in \vref{fig:schema-graph}
			
			\begin{figure}
				\centering
				
				\begin{dot2tex}[neato]
					digraph G {
						node [shape=plaintext]; Subject; Course; Term; Section; Schedule; Instructor;
						
						Course -> Subject;
						Section -> Term
						Section -> Course;
						Schedule -> Section;
						Schedule -> Instructor;
					}
				\end{dot2tex}
				
				\caption{Graph representation of relations (\cref{fig:schema}) and \gls{fk} (\cref{eqn:fk-constraints})}
				\label{fig:schema-graph}
			\end{figure}
		\end{ex}
		
		The relational data model is particularly powerful for analytic queries.  Given the schema graph in \vref{fig:schema-graph}, one can formulate the following analytic queries in a query language known as \gls{sql}.
			
		\begin{ex}
			Using \gls{sql}, find all section CRNs for the subject titled ``Community Development \& Policy Studies.''
			
			\begin{figure}
				\begin{singlespaced}
					\begin{pygments}{sql}
SELECT section.crn
FROM   section 
       JOIN course 
         ON section.course_code = course.code 
       JOIN subject 
         ON subject.id = course.subject_id 
WHERE  subject.name = 'Community Development & Policy Studies'; 
					\end{pygments}
				\end{singlespaced}
				
				\caption{Query to find section CRNs for a subject name.}
				\label{fig:query-section-crns}
			\end{figure}
			
			The \gls{sql} query in \vref{fig:query-section-crns} results in \vref{tbl:query-section-crns-output}.
			
			\begin{table}
				\centering
				
				\begin{tabular}{l}
					\toprule
					crn \\
					\midrule
					10000 \\
					10001 \\
					10002 \\
					\bottomrule
				\end{tabular}
				
				\caption{Results of the query in \vref{fig:query-section-crns}.}
				\label{tbl:query-section-crns-output}
			\end{table}
		\end{ex}
		
	\subsection{Entity Group}
		\begin{defn}[Entity Group]
		\label{def:entity-group}
			An entity group is a forest, \(\egraph\), of tuples interconnected by join conditions defined by the \gls{fk} constraints in the schema graph \(\sgraph\).
			
			Given two vertices \(\tuple, \tuple' \in \text{V}(\egraph)\), \(\exists \relation_1, \relation_2 \in \relations{\db}\) such that \(\tuple \in \relation_1\), \(\tuple'\in \relation_2\), and \((\relation_1, \relation_2)\in \sgraph\).  That is, \(\tuple\) and \(\tuple'\) belong to two relations that are connected by the schema graph.

			Let \(\relation_1(\attribute_{11}, \attribute_{12}, \dotsc, \attribute_{1k}) \to \relation_2(\attribute_{21}, \attribute_{22}, \dotsc, \attribute_{2k})\) be the \gls{fk} that connects \(\relation_1, \relation_2\).  We further assert that \(\tuple\lbrack\attribute_{11}, \attribute_{12}, \dotsc, \attribute_{1k}\rbrack = \tuple'\lbrack\attribute_{21}, \attribute_{22}, \dotsc, \attribute_{2k}\rbrack\).
		\end{defn}
		
		Entity groups define complex, structured objects that include more information than individual tuples in the relations.
		
		\begin{ex}
			The information in \vref{tbl:hmr-properties} all relates to the Course titled Human-Mutant Relations, however no single tuple in the database has all of this information as a result of database normalization.
			
			\begin{table}
				\centering
				
				\begin{tabular}{ll}
					\toprule
					Attribute & Value \\
					\midrule
					code & CDPS 101 \\
					title & Human-Mutant Relations \\
					subject & Community Development \& Policy Studies \\
					\bottomrule
				\end{tabular}
				
				\caption{Properties of the Course titled Human-Mutant Relations.}
				\label{tbl:hmr-properties}
			\end{table}
			
			We require an entity group (\vref{fig:hmr-entity-group}) to join together all pieces of information related to this course. 
			
			\begin{figure}
				\centering
				
				\begin{dot2tex}[dot]
digraph G {
	node [shape=plaintext]; "Human-Mutant Relations"; "Community Development & Policy Studies";
	
	"Human-Mutant Relations" -> "Community Development & Policy Studies";
}
				\end{dot2tex}
				
				\caption{Human-Mutant Relations entity group}
				\label{fig:hmr-entity-group}
			\end{figure}

		\end{ex}

	\subsection{Pros and Cons of the Relational Model}
		In order to better understand the motivation behind this work, it is important to examine both the strong and weak points of the relational model.
		
		\subsubsection{Pros}
			The enforcement of constraints is essential to the relational model.  There are several types of constraints, including uniqueness and \glspl{fk}.  The first constraint maintains uniqueness.
			
			The Course relation (\vref{tbl:course-relation}) has the attribute \texttt{code} as its primary key.	In order for other relations to reference a specific named tuple, the \texttt{code} attribute must be unique.
			
			\begin{ex}[Unique Constraint]
			\label{ex:unique-constraint}
				Attempt to insert another course with a \texttt{code} of ``CDPS 101.''
				
				\begin{singlespaced}
					\begin{pygments}{sql}
INSERT INTO course
VALUES      ('CDPS 101',
             'Mutant-Human Relations',
             'CDPS');
					\end{pygments}
				\end{singlespaced}
				
				The \gls{rdbms} enforces the primary key constraint on the \texttt{code} attribute, rejecting the insertion.
				
				\begin{verbatim}
Error: column code is not unique
				\end{verbatim}
			\end{ex}
			
			With the uniqueness of named tuples guaranteed (as demonstrated in \vref{ex:unique-constraint}), we must ensure that any named tuples that are referenced actually exist.  If they do not, the database must not permit the operation to continue.  Doing so would lead to dangling references.
			
			\begin{ex}[Referential Integrity]
				Attempt to insert the tuple (``CHEM 101'', ``Introductory Chemistry'', ``'CHEM'') in the Course relation.
				
				\begin{singlespaced}
					\begin{pygments}{sql}
INSERT INTO course
VALUES      ('CHEM 101',
             'Introductory Chemistry',
             'CHEM');
					\end{pygments}
				\end{singlespaced}
				
				Again we see the \gls{rdbms} protecting the integrity of the data.
				
				\begin{verbatim}
Error: foreign key constraint failed
				\end{verbatim}
			\end{ex}
			
			In addition to enforcing consistency, the relational model is capable of providing higher-level views of the data through aggregation.
			
			\begin{ex}[Aggregation]
				Find the number of sections offered for the subject named ``Community Development \& Policy Studies.''
				
				\begin{singlespaced}
					\begin{pygments}{sql}
SELECT Count(*)
FROM   section
       JOIN course
         ON section.course = course.code
       JOIN subject
         ON subject.id = course.subject
WHERE  subject.name = 'Community Development & Policy Studies';
					\end{pygments}
				\end{singlespaced}
			\end{ex}
			
			Information stored within a properly designed database is normalized.  That is, no information is repeated.
			
			\begin{ex}[Normalization]
				For example, suppose Emma Frost became headmistress and the subject named ``Community Development \& Policy Studies'' was renamed to ``Community Destruction \& Policy Studies.''  If this information were not normalized, each course in this subject would need to be updated.  Since this information is normalized, the following query will suffice.
				
				\begin{singlespaced}
					\begin{pygments}{sql}
UPDATE subject
SET    name = 'Community Destruction & Policy Studies'
WHERE  id = 'CDPS';
					\end{pygments}
				\end{singlespaced}
			\end{ex}
			
			The above examples are some of the most important reasons for choosing the relational model over others.	Unfortunately, the relational model is not without its downsides.
		
		\subsubsection{Cons}
			While the relational model excels at ensuring data consistency, aggregation, and reporting; it is not suitable for every task.	In order to issue queries, a user must be familiar with the schema.	 This requires specific domain knowledge of the data.
			
			An example of a complicated query involving two joins is give in \vref{fig:query-section-crns}.
			
			A casual user is unlikely to determine the correct join path, name of the tables, name of the attributes, etc.	This is in contrast to the document model, where the data is semi-structured or unstructured, requiring minimal domain knowledge.

			The relational model is also rigid in structure.  If a relation is modified, every query referencing said relation may require a rewrite.  Even a simple attribute being renamed (e.g.~\(\rho_{\text{name/alias}}(\text{Person})\)) is capable of modifying the join paths.  This rigidity places additional cognitive burden on users.
			
			In addition to having a rigid structure, most relational database management systems lack flexible string matching options.	 Assuming basic SQL-92 compliance, a \gls{rdbms} only supports the \texttt{LIKE} predicate \cite{sql-2011}.
			
			\begin{ex}[\texttt{LIKE} Predicate]
				Find all courses with a title that contains ``man.''
				
				\begin{singlespaced}
					\begin{pygments}{sql}
SELECT *
FROM   course
WHERE  title LIKE '%man%';
					\end{pygments}
				\end{singlespaced}
			\end{ex}
			
			There are a couple of limitations to the \texttt{LIKE} predicate.  First, it only supports basic substring matching.  If a user accidentally searches for all courses with a title containing ``men,'' nothing would be found.
			
			Second, unless the predicate is applied to the end of the string and the column is indexed, performance will be poor.  The database must scan the entire table in order to answer the query, resulting in performance of \(\mathcal{O}(n)\), where \(n\) is the number of named tuples in the relation.
% !TEX root = Thesis.tex
\section{Document Model}
\label{sec:document-model}
	In this section we formally define the document model.
	
	Documents are a unit of information.  The definition of unit can vary.  It may represent an email, a book chapter, a memo, etc.  Contained within each document is a set of terms.
	
	In contrast to the relational model, the document model represents semi-structured as well as unstructured data.  Examples of information suitable to the document model includes emails, memos, book chapters, etc.
	
	These pieces, or units, of information are broken into documents.  Groups of related documents (for example, a library catalogue) are referred to as a document collection.

	\begin{defn}[Terms and Document]
	\label{def:document}
		A term, $t$, is an indivisible string (e.g.~a proper noun, word, or a phrase).  A document, $d$, is a bag of words.  Let $\freq\left(t, d\right)$ be the frequency of terms $t$ in document $d$.
		
		Let $T$ denote all possible terms, and $\fcn{Bag}{T}$ be all possible bag of terms.
	\end{defn}
	
	\begin{remark}
		We use the bag-of-words model for documents.  This means that position information of terms in a document is irrelevant, but the frequency of terms are kept in the document.  Documents are non-distinct sets.
	\end{remark}
	
	\begin{defn}[Document Collection]
	\label{def:document-collection}
		A document collection $D$ is a set of documents, written $D = \left\{d_1, d_2, \dotsc, d_k\right\}$.  The size of $D$ is denoted $\gls{ndocs}$.  The number of unique terms, or size of $\gls{terms}$, in $D$, is denoted $\gls{nterms}$.
	\end{defn}
	
	\begin{ex}
	\label{ex:superhero-documents}
		Consider the following short sentences.
		
		\begin{enumerate}
			\item Superman is strong on Earth and lives on Earth.
			\item Batman was born on Earth.
			\item Superwoman is fast on Earth.
			\item Superman was born on Krypton.
		\end{enumerate}
		
		Each sentence represents a document, giving us the following documents.
		
		\begin{eqnarray*}
			d_1 &=& \left\{\textrm{``and''}: 1, \textrm{``on''}: 2, \textrm{``is''}: 1, \textrm{``lives''}: 1, \textrm{``earth''}: 2, \textrm{``strong''}: 1, \textrm{``superman''}: 1\right\} \\
			d_2 &=& \left\{\textrm{``batman''}: 1, \textrm{``on''}: 1, \textrm{``was''}: 1, \textrm{``earth''}: 1, \textrm{``born''}: 1\right\} \\
			d_3 &=& \left\{\textrm{``on''}: 1, \textrm{``is''}: 1, \textrm{``superwoman''}: 1, \textrm{``fast''}: 1, \textrm{``earth''}: 1\right\} \\
			d_4 &=& \left\{\textrm{``krypton''}: 1, \textrm{``born''}: 1, \textrm{``on''}: 1, \textrm{``was''}: 1, \textrm{``superman''}: 1\right\} \\
		\end{eqnarray*}
	\end{ex}
	
	\subsection{Vectorization of Documents}
	\label{sec:vectorization-of-documents}
		One of the most fundamental approach for search documents is to treat documents as high dimensional vectors, and the document collection as a subset in a vector space.  The search query becomes a nearest neighbour query in a vector space equipped with a distance measure.
		
		The first step is to convert bag of terms into vectors.  The standard technique \cite{ir-08} uses a scoring function that measures the relative importance terms in documents.
		
		\begin{defn}[TF-IDF Score]
			The term frequency is the number of times a term $t$ appears in a document $d$, as given by $\freq\left(t, d\right)$.  The document frequency of a term $t$, denoted by $\df_t$, is the number of documents in $D$ that contains $t$.  It is defined as
			
			$$\df_t = \mid \left\{d \in D: t \in d\right\} \mid$$
			
			The combined TF-IDF score of $t$ in a document $d$ is given by
			
			$$\tfidf\left(D, t, d\right) = \frac{\freq\left(t, d\right)}{\mid d \mid} \cdot \log{\frac{N}{\df_t}}$$
		\end{defn}
		
		\begin{remark}
			The first component, $\frac{\freq\left(t, d\right)}{\mid d \mid}$, measures the importance of a term within a document.  It is normalized to account for document length.  The second component, $\log{\frac{N}{\df_t}}$, is a measure of the rarity of the term within the document collection $D$.
		\end{remark}
		
		\begin{ex}
			Using the documents from Example~\ref{ex:superhero-documents}, the TF-IDF scores are as follows.
			
			$$\bordermatrix{
				~ & d_1 & d_2 & d_3 & d_4 \cr
				t_1 : \textrm{``and''} & 0.2857 & 0.0000 & 0.0000 & 0.0000 \cr
				t_2 : \textrm{``on''} & 0.0000 & 0.0000 & 0.0000 & 0.0000 \cr
				t_3 : \textrm{``superwoman''} & 0.0000 & 0.0000 & 0.4000 & 0.0000 \cr
				t_4 : \textrm{``batman''} & 0.0000 & 0.4000 & 0.0000 & 0.0000 \cr
				t_5 : \textrm{``is''} & 0.1429 & 0.0000 & 0.2000 & 0.0000 \cr
				t_6 : \textrm{``fast''} & 0.0000 & 0.0000 & 0.4000 & 0.0000 \cr
				t_7 : \textrm{``born''} & 0.0000 & 0.2000 & 0.0000 & 0.2000 \cr
				t_8 : \textrm{``krypton''} & 0.0000 & 0.0000 & 0.0000 & 0.4000 \cr
				t_9 : \textrm{``earth''} & 0.1186 & 0.0830 & 0.0830 & 0.0000 \cr
				t_{10} : \textrm{``lives''} & 0.2857 & 0.0000 & 0.0000 & 0.0000 \cr
				t_{11} : \textrm{``strong''} & 0.2857 & 0.0000 & 0.0000 & 0.0000 \cr
				t_{12} : \textrm{``was''} & 0.0000 & 0.2000 & 0.0000 & 0.2000 \cr
				t_{13} : \textrm{``superman''} & 0.1429 & 0.0000 & 0.0000 & 0.2000 \cr
			}$$
		\end{ex}

		\begin{defn}[Document Vector]
			Given a document collection $D$ with $M$ unique terms $T = \left[ t_1, t_2, \dotsc, t_n \right]$, each document $d$ can be represented by an $M$-dimensional vector.
			
			$$
				\vec{d} = 
				\left[
				\begin{array}{c}
					\tfidf\left(t_1, d\right) \\
					\tfidf\left(t_2, d\right) \\
					\vdots \\
					\tfidf\left(t_n, d\right)
				\end{array}
				\right]
			$$
		\end{defn}
		
		\begin{ex}
			The documents in Example~\ref{ex:superhero-documents} would produce the following vectors.
			
			$$
			\vec{d_n} = 
				\left[
					\begin{array}{l}
						\tfidf\left(t_{1}, d_n\right) \\
						\tfidf\left(t_{2}, d_n\right) \\
						\tfidf\left(t_{3}, d_n\right) \\
						\tfidf\left(t_{4}, d_n\right) \\
						\tfidf\left(t_{5}, d_n\right) \\
						\tfidf\left(t_{6}, d_n\right) \\
						\tfidf\left(t_{7}, d_n\right) \\
						\tfidf\left(t_{8}, d_n\right) \\
						\tfidf\left(t_{9}, d_n\right) \\
						\tfidf\left(t_{10}, d_n\right) \\
						\tfidf\left(t_{11}, d_n\right) \\
						\tfidf\left(t_{12}, d_n\right) \\
						\tfidf\left(t_{13}, d_n\right) \\
					\end{array}
				\right],
			\vec{d_1} = 
				\left[
					\begin{array}{l}
						0.2857 \\
						0.0000 \\
						0.0000 \\
						0.0000 \\
						0.1429 \\
						0.0000 \\
						0.0000 \\
						0.0000 \\
						0.1186 \\
						0.2857 \\
						0.2857 \\
						0.0000 \\
						0.1429 \\
					\end{array}
				\right],
			\vec{d_2} = 
				\left[
					\begin{array}{l}
						0.0000 \\
						0.0000 \\
						0.0000 \\
						0.4000 \\
						0.0000 \\
						0.0000 \\
						0.2000 \\
						0.0000 \\
						0.0830 \\
						0.0000 \\
						0.0000 \\
						0.2000 \\
						0.0000 \\
					\end{array}
				\right],
			\vec{d_3} = 
				\left[
					\begin{array}{l}
						0.0000 \\
						0.0000 \\
						0.4000 \\
						0.0000 \\
						0.2000 \\
						0.4000 \\
						0.0000 \\
						0.0000 \\
						0.0830 \\
						0.0000 \\
						0.0000 \\
						0.0000 \\
						0.0000 \\
					\end{array}
				\right],
			\vec{d_4} = 
				\left[
					\begin{array}{l}
						0.0000 \\
						0.0000 \\
						0.0000 \\
						0.0000 \\
						0.0000 \\
						0.0000 \\
						0.2000 \\
						0.4000 \\
						0.0000 \\
						0.0000 \\
						0.0000 \\
						0.2000 \\
						0.2000 \\
					\end{array}
				\right]
			$$
		\end{ex}
		
		\begin{defn}[Search Query]
			A search query $q$ is simply a document, namely a bag of terms.  The top-$k$ answers to $q$ with respect to a collection $D$ is defined as the $k$ documents, $\left\{d_1, d_2, \dotsc, d_k\right\}$, in $D$, such that $\left\{\vec{d}_i\right\}$ are the closest vectors to $\vec{q}$ using Euclidean distance measure in $\mathbb{R}^N$.
		\end{defn}
		
		\begin{ex}
			Given the search query $q = \left\{ \mathrm{superwoman}, \mathrm{was}, \mathrm{born}, \mathrm{on}, \mathrm{krypton} \right\}$, compute the vector $\vec{q}$ within the document collection $D$ (as defined in Example~\ref{ex:superhero-documents}).
			
			$$
			\vec{q} = 
				\left[
					\begin{array}{c}
						0.0000 \\
						0.0000 \\
						0.0000 \\
						0.0000 \\
						0.0000 \\
						0.0000 \\
						0.0000 \\
						0.1474 \\
						0.2644 \\
						0.0000 \\
						0.2644 \\
						0.1474 \\
						0.0000 \\
					\end{array}
				\right]
			$$
		\end{ex}
		
		In order to determine the top-$k$ documents for search query $q$, we need a way of measuring the similarity between documents.
		
		\begin{defn}[Cosine Similarity]
			Given two document vectors, $\vec{d}_1$ and $\vec{d}_2$, the cosine similarity is the dot product $\vec{d}_1 \cdot \vec{d}_2$, normalized by the product of the Euclidean distance of $\vec{d}_1$ and $\vec{d}_2$ in $\mathbb{R}^N$.  It is denoted as $\similarity\left(\vec{d}_1, \vec{d}_2\right)$.
			
			\begin{eqnarray}
				\similarity\left(\vec{d}_1, \vec{d}_2\right) &=& \frac{\vec{d}_1 \cdot \vec{d}_2}{\mid\mid \vec{d}_1 \mid\mid \cdot \mid\mid \vec{d}_2 \mid\mid} \\
				 &=& \frac{\sum\limits_{i=1}^{N} \vec{d}_{1, i} \times \vec{d}_{2, i}}{\sqrt{\sum\limits_{i=1}^{N} \left(\vec{d}_{1, i}\right)^2} \times \sqrt{\sum\limits_{i=1}^{N} \left(\vec{d}_{2, i}\right)^2}}
			\end{eqnarray}
		\end{defn}
		
		Recall we may represent search queries as documents and thus document vectors.  Therefore we may compute the score of a document $d$ for a search query $q$ as
		
		$$\similarity\left(\vec{d}, \vec{q}\right)$$
		
		\begin{ex}
			Given the document collection $D$ (from Example~\ref{ex:superhero-documents}) and search query $q$, compute the similarity between $q$ and every document $d \in D$.
			
			\begin{eqnarray}
				\similarity\left(\vec{d}_1, \vec{q}\right) &=& 0.000000 \\
				\similarity\left(\vec{d}_2, \vec{q}\right) &=& 0.191533 \\
				\similarity\left(\vec{d}_3, \vec{q}\right) &=& 0.265877 \\
				\similarity\left(\vec{d}_4, \vec{q}\right) &=& 0.618553
			\end{eqnarray}
		\end{ex}
		
	\subsection{Extending the Document Model}
	\label{sec:extending-the-document-model}
		In the extended document model, documents have attributes: $\fcn{ATTR}{d}$, and each attribute have values (e.g.~date, string, integer), or bag of terms.  Thus:

		$$d:\fcn{ATTR}{d} \to \fcn{BAG}{\mathrm{Terms}}$$
		
		\begin{ex}[Semi-Structured Document]
			We see that $d_2$ is about Batman.  The document contents are semi-structured, containing both a name and the name of a planet.  By adding attributes to the document, we are left with Table~\ref{tbl:person-document}.
			
			\begin{table}[!ht]
				\centering
				
				\begin{tabular}{ll}
					\toprule
					Attribute & Value \\
					\midrule
					name & Batman \\
					birthplace & Earth \\
					body & Batman was born on Earth. \\
					\bottomrule
				\end{tabular}
				
				\caption{Person document for Batman}
				\label{tbl:person-document}
			\end{table}
			
			which is similar in structure to the \texttt{Person} table.
		\end{ex}
		
	\subsection{Approximate String matching}
	\label{sec:n-gram}
		\begin{defn}[N-Gram]
			An $n$-Gram is a contiguous sequence of substrings of string $S$ of length $n$.  An algorithm for computing the $n$-gram of $S$ is given in Algorithm~\ref{alg:n-gram}.
		\end{defn}
		
		% \char"24 - DOLLAR  BILL Y'ALL

		\begin{algorithm}[!ht]
			\caption{$\textsc{N-Gram}\left(S, n, s\right)$}
			\label{alg:n-gram}
			
			\begin{singlespaced}
				\begin{algorithmic}[1]
					\REQUIRE $S$ is a string, $n \ge 1$, and $s$ is a character
					\ENSURE the list of $n$-grams of $S$
					\medskip
					\STATE $p \leftarrow \textsc{Repeat}\left(s, n - 1\right)$\label{alg:n-gram:repeat}\label{alg:n-gram:p}
					\STATE $S \leftarrow \textsc{Replace}\left(S, \mathrm{'\;'}, \mathrm{'\char"24'}\right)$\label{alg:n-gram:replace}
					\STATE $S \leftarrow \textsc{Concat}\left(p, S, p\right)$\label{alg:n-gram:concat}\label{alg:n-gram:S}
					\STATE $G \leftarrow \left[\right]$\label{alg:n-gram:G}
					\STATE $l \leftarrow \textsc{Length}\left(S\right)$\label{alg:n-gram:length}
					
					\FOR{$i=0$ \TO $l - n + 1$}
						\STATE $G \leftarrow \textsc{SubStr}(S, i, i + n)$
					\ENDFOR
					
					\RETURN $G$
					\medskip
					\medskip
				\end{algorithmic}
			\end{singlespaced}
		\end{algorithm}
		
		\todo{The medskips above should not be required.}
		
		Where,
		 
		\begin{itemize}
			\item $\textsc{Repeat}\left(c, i)\right)$ returns a string with character $c$ repeated $i$ times (line \ref{alg:n-gram:repeat}),
			\item $\textsc{Replace}\left(s, c_1, c_2\right)$ returns string $s$ with all instances of character $c_1$ replaced by $c_2$ (line \ref{alg:n-gram:replace}),
			\item $\textsc{Concat}\left(s_1, s_2, \dotsc, s_n\right)$ returns the concatenation of $s_1, s_2, \dotsc, s_n$ (line \ref{alg:n-gram:concat}),
			\item $\textsc{Length}\left(s\right)$ returns the number of characters in string $s$ (line \ref{alg:n-gram:length}),
			\item and $\textsc{SubStr}\left(s, i_1, i_2\right)$ returns a substring of string $s$ from $i$ (inclusive) to $i + n$ (exclusive).
		\end{itemize}
		
		\begin{ex}
		\label{ex:ngram-banana}
			Given a string $S = \mathrm{``banana''}$, compute the trigram of $S$ using Algorithm \ref{alg:n-gram}.  Lines \ref{alg:n-gram:p}-\ref{alg:n-gram:S} would yield $S = \mathrm{``\char"24\char"24banana\char"24\char"24''}$, resulting in $l = 10$ (line \ref{alg:n-gram:length}).
			
			$$
				G = \left\{
					\mathrm{``\char"24\char"24b''},
					\mathrm{``\char"24ba''},
					\mathrm{``ban''},
					\mathrm{``ana''},
					\mathrm{``nan''},
					\mathrm{``ana''},
					\mathrm{``na\char"24''},
					\mathrm{``a\char"24\char"24''}
				\right\}
			$$
			
			With a frequency of terms of
			
			$$\left\{\mathrm{``ana''}: 2, \mathrm{``na\char"24''}: 1, \mathrm{``\char"24ba''}: 1, \mathrm{``nan''}: 1, \mathrm{``a\char"24\char"24''}: 1, \mathrm{``ban''}: 1, \mathrm{``\char"24\char"24b''}: 1\right\}$$
		\end{ex}
		
		We use $n$-grams in order to permit approximate string matching.
		
		\begin{ex}
		\label{ex:n-gram-comparison}
			Given a string $S$ (Example~\ref{ex:ngram-banana}), let $S' = \mathrm{``vanana''}$.  Compute the trigram of $S'$ and compare it to $S$.
			
			$$
				G' = \left\{
					\mathrm{``\char"24\char"24v''},
					\mathrm{``\char"24va''},
					\mathrm{``van''},
					\mathrm{``ana''},
					\mathrm{``nan''},
					\mathrm{``ana''},
					\mathrm{``na\char"24''},
					\mathrm{``a\char"24\char"24''}
				\right\}
			$$
			
			Comparing $G$ to $G'$ results in the following matrix
			
			\begin{figure}[!ht]
				$$
					\bordermatrix{
					~ & t_1 & t_2 & t_3 & t_4 & t_5 & t_6 & t_7 & t_8 \cr
					G & \mathrm{``\char"24\char"24b''} & \mathrm{``\char"24ba''} & \mathrm{``ban''} & \mathrm{``ana''} & \mathrm{``nan''} & \mathrm{``ana''} & \mathrm{``na\char"24''} & \mathrm{``a\char"24\char"24''} \cr
					G' & \mathrm{``\char"24\char"24v''} & \mathrm{``\char"24va''} & \mathrm{``van''} & \mathrm{``ana''} & \mathrm{``nan''} & \mathrm{``ana''} & \mathrm{``na\char"24''} & \mathrm{``a\char"24\char"24''} \cr
					}
				$$
				
				\caption{Comparison between $n$-grams of $G$ and $G'$.}
				\label{fig:n-gram-misspelling-comparison}
			\end{figure}
			
			As Figure~\ref{fig:n-gram-misspelling-comparison} shows, using $n$-grams yield very similar documents despite the letter substitution.
		\end{ex}
			
	\subsection{Pros and Cons of the Document Model}
		There are numerous reasons to use the document model.  It allows users without domain knowledge and working knowledge of a complex query language such as \gls{sql} to find information.
		
		\begin{ex}[Simple Queries]
			Find all documents related to ``Superman'' or ``Earth''.  This query, if the default operator is \texttt{OR}, would simply be \texttt{Superman Earth}.  The result of the query $q$ would be
			
			$$\texttt{superman earth} = \left\{d_1, d_2, d_3, d_4\right\}$$
		\end{ex}
		
		Users can also modify queries to require certain terms be present or not present.
		
		\begin{ex}[\texttt{AND} Query]
		\label{ex:and-query}
			Find all documents containing both ``Superman'' and ``Earth''.  This query would return the following set of documents
			
			$$\texttt{superman AND earth} = \left\{d_1\right\}$$
			
			as only $d_1$ contains both terms.
		\end{ex}
		
		\begin{ex}[\texttt{NOT} Query]
			Find all documents containing ``Superman'' but not ``Earth''.  This query would return different results than Example~\ref{ex:and-query}.
			
			$$\texttt{superman NOT earth} = \left\{d_4\right\}$$
		\end{ex}
		
		While none of the above queries required domain knowledge, it is possible to use the extended document model (Section~\ref{sec:extending-the-document-model}) to search specific fields.  Doing so allows users to have finer control over what documents are retrieved.
		
		\begin{ex}[Extended Query]
			Find all documents with a superhero named ``Superman'' that contain the term ``Earth''.
			
			$$\texttt{name:Superman Earth} = \left\{d_1\right\}$$
			
			Assuming the first term of every document is also the value of the name attribute.
		\end{ex}
		
		People utilize keyword query search every day through web search engines such as Google\footnote{\url{https://www.google.ca/}}.
		
		Not only does the document model provide a familiar interface to search for information with, it also ranks the results.  In the relational model a search for ``Superman'' would return all named tuples that contained that term.  In the document model, documents are ranked against the query $q$ and the top-$k$ documents are returned.
		
		The advantage is that users have the result of $q$ already ranked so only the most relevant documents may be explored.  As the number of documents matching $q$ for a large corpus can be high, showing only the top-$k$ relevant documents may save the user a substantial amount of time.
		
		The relational model does not permit approximate string matching.  By utilizing the document model with $n$-grams (Section~\ref{sec:n-gram}), users who substitute, delete, or insert characters from the desired term may still receive results for their intended term (see Example~\ref{ex:n-gram-comparison} for a demonstration of how $n$-grams overcome character substitutions).
		
		Unfortunately the document model does not support the concept of foreign keys (Definition~\ref{def:foreign-keys}).  While information is easily accessible due to flexible search, each document is a discrete unit of information.  Aggregate queries are unsupported, as these units are not linked amongst one another.