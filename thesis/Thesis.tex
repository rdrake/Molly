% Change "draft" to "final" when you're done.
\documentclass[12pt,letterpaper,oneside,draft]{report}

\usepackage[top=1in, left=1.5in, right=1in, bottom=1in]{geometry}

\usepackage{fontspec,lipsum}
\usepackage{mathpazo}
\defaultfontfeatures{Ligatures=TeX}
\usepackage[small,sf,bf]{titlesec}

\usepackage{xltxtra}
\usepackage{xunicode}

%\DisemulatePackage{setspace}
\usepackage{setspace}
\onehalfspacing

\usepackage{titling}

\usepackage{unicode-math}
\usepackage{amsmath}
\usepackage{amsthm}

\usepackage{algorithm}
\usepackage{algorithmic}

%\usepackage{minted}
\usepackage{tikz}

\usepackage{nag}

\setmainfont[
	Numbers=OldStyle,
	Kerning=Uppercase,
	SizeFeatures={
		{Size={8-10}, Font=* Caption},
		{Size={10.01-}, Font=*}
	}]{Garamond Premier Pro}
\setsansfont{Futura Std}
\setmathfont{STIXGeneral}
\setmonofont{Consolas}

\pagenumbering{roman}

\newcommand{\thesistitle}[8]{
	\title{#1}
	\author{#2}

	\begin{titlepage}
		\centering
		\vspace{2cm}
			{\huge\sc #1}\\[1cm]

			by\\[1cm]

			{\Large #2}\\[2cm]

			\singlespacing{A Thesis Submitted in Partial Fulfillment\\of the Requirements for the Degree of}\\[1cm]

			#3\\[0.5cm]

			in\\[0.5cm]

			Faculty of #4\\[0.5cm]

			#5\\[1.5cm]

			University of Ontario Institute of Technology\\[1.5cm]

			Supervisor:  #6\\[1.5cm]

			#7 #8\\[3cm]

			Copyright \copyright\ #2 #8
	\end{titlepage}}

\theoremstyle{definition}
\newtheorem{defn}{Definition}

\begin{document}
	\thesistitle{Molly}{Richard Drake}{Masters of Science}{Science}{Computer Science}{Dr. Ken Q. Pu}{September}{2012}

	\begin{abstract}
		\lipsum[2]
	\end{abstract}

%	\begin{acknowledgements}
%		\lipsum[2]
%	\end{acknowledgements}

	\tableofcontents

	\listoftables
	\listoffigures
	\listofalgorithms

	\clearpage
	\pagenumbering{arabic}

	\chapter{Introduction}

		\section{Outline}

	\chapter{Literature Review}

	\chapter{System Design, Rationale, \& Performance}
		\section{Introduction}
			\subsection{Data Representation}
				The data from the database is represented in various data structures.  There are separate representations for each type of data:  values, entities, and entity groups.

				\subsubsection{Value}
					\begin{defn}
						A \textbf{Value} represents a single piece of information.  To avoid repetition, each value is unique.  That is, $\exists! v \in V$, where $v$ is a value in the set $V$ of all values.
					\end{defn}

				\subsubsection{Entity}
					\begin{defn}
						An \textbf{Entity} is a collection of attributes, $a_n$, each mapped to a single value, $v_n$.  An entity also includes additional information such as a unique identifier.

						\begin{figure}[!ht]
							\centering
							\[
								\begin{array}{ll}
									\mathrm{id} & T_n|v_{id} \\
									a_1 & v_1 \\
									a_2 & v_2 \\
									\vdots & \vdots \\
									a_n & v_n
								\end{array}
							\]
							\caption{The structure of an entity}
							\label{fig:entity-rep}
						\end{figure}

						Entities are analogous to rows in a database table.  Thus, the unique identifier is generated based on the table name, $T_n$, as well as unique key in the table, $v_{id}$.  The unique key identifies the row, and the table name identifies the table.  Together they uniquely identify the entity within the entire database.

						$\exists! e_{id} \in E$, where $E$ is the set of all entities.
					\end{defn}

				\subsubsection{Entity Group}
					\begin{defn}
						An \textbf{Entity Group} joins together two or more entities.  These entity groups can also have attributes, $a_n$, and values, $v_n$, associated with them much like entities.

						\begin{figure}[!ht]
							\centering
							\[
								\begin{array}{ll}
									e_L & \left[ e_1, e_2, \ldots, e_n \right] \\
									a_1 & v_1 \\
									a_2 & v_2 \\
									\vdots & \vdots \\
									a_n & v_n
								\end{array}
							\]
							\caption{The structure of an entity group}
							\label{fig:entity-group-rep}
						\end{figure}

						\begin{tikzpicture}
							[scale=.8,auto=left,every node/.style={circle}]
							\node (e1) at (1,1) {$e_1$};
							\node (e2) at (4,2)  {$e_2$};
							\node (e3) at (7,3)  {$e_3$};
							\node (e4) at (10,3) {$e_4$};
							\node (e5) at (13,2)  {$e_5$};
							\node (e6) at (16,1)  {$e_6$};

							\foreach \from/\to in {e1/e2,e2/e3,e3/e4,e4/e5,e5/e6}
								\draw (\from) -- (\to);

						\end{tikzpicture}
					\end{defn}

			\section{Overview of System Components}
				\subsection{Crawler}
				\subsection{Indexer}
				\subsection{Core (Processing)}
				\subsection{API}
				\subsection{Frontend}

			\section{Data Flow}
			
			\section{Implementation Issues}

		\section{Ford-Fulkerson}
			\begin{algorithm}
				\begin{algorithmic}
					\ENSURE $1=1$
				\end{algorithmic}
			\end{algorithm}

	
	\chapter{Blah blah blah}
		\section{Choosing a Breadth-First Search Algorithm}
			We took several criteria into consideration when choosing a Breadth-First Search algorithm to perform the graph search.  In the context of this problem, there is no obvious heuristic to help predict the distance between two nodes.  We decided to use a constant cost function.  That is, $d_{(v_1, v_2)} = n$.  The selected algorithm must also be highly parallelizable.

			

			BFS - Wrong result if non-uniform distance
			Bellman-Ford - Always correct result
			Dijkstra - Faster than Bellman-Ford, uses priority queue
			Push-Relabel \& Johnson - Uses complicated (and difficult to parallelize) data structures
			A-Star Search - No obvious heuristic
	
	
\end{document}
