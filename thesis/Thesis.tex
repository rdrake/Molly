% Change "draft" to "final" when you're done.
\documentclass[12pt,letterpaper,oneside,draft]{report}

\usepackage[top=1in, left=1.5in, right=1in, bottom=1in]{geometry}

\usepackage{fontspec,lipsum}
\usepackage{mathpazo}
\defaultfontfeatures{Ligatures=TeX}
\usepackage[small,sf,bf]{titlesec}

%\DisemulatePackage{setspace}
\usepackage{setspace}
\onehalfspacing

\usepackage{titling}

\usepackage{unicode-math}
\usepackage{amsmath}
\usepackage{amsthm}

\usepackage{algorithm}
\usepackage{algorithmic}

\setmainfont[
	Numbers=OldStyle,
	Kerning=Uppercase,
	SizeFeatures={
		{Size={8-10}, Font=* Caption},
		{Size={10.01-}, Font=*}
	}]{Garamond Premier Pro}
\setsansfont{Futura Std}
\setmathfont{STIXGeneral}

\pagenumbering{roman}

\newcommand{\thesistitle}[8]{
	\title{#1}
	\author{#2}

	\begin{titlepage}
		\centering
		\vspace{2cm}
			{\huge\sc #1}\\[1cm]

			by\\[1cm]

			{\Large #2}\\[2cm]

			\singlespacing{A Thesis Submitted in Partial Fulfillment\\of the Requirements for the Degree of}\\[1cm]

			#3\\[0.5cm]

			in\\[0.5cm]

			Faculty of #4\\[0.5cm]

			#5\\[1.5cm]

			University of Ontario Institute of Technology\\[1.5cm]

			Supervisor:  #6\\[1.5cm]

			#7 #8\\[3cm]

			Copyright \copyright\ #2 #8
	\end{titlepage}}

\theoremstyle{definition}
\newtheorem{defn}{Definition}

\begin{document}
	\thesistitle{Molly}{Richard Drake}{Masters of Science}{Science}{Computer Science}{Dr. Ken Q. Pu}{September}{2012}

	\begin{abstract}
		\lipsum[2]
	\end{abstract}

%	\begin{acknowledgements}
%		\lipsum[2]
%	\end{acknowledgements}

	\tableofcontents
	\listoffigures
	\listoftables

	\clearpage
	\pagenumbering{arabic}

	\chapter{Introduction}

		\section{Data Representation}
			The data from the database is represented in various data structures.  There are separate representations for each type of data:  values, entities, and entity groups.

			\subsection{Value}
				\begin{defn}
					A \textbf{Value} represents a single piece of information.  To avoid repetition, each value is unique.  That is, $\exists! v \in V$, where $v$ is a value in the set $V$ of all values.
				\end{defn}

			\subsection{Entity}
				\begin{defn}
					An \textbf{Entity} is a collection of attributes, $a_n$, each mapped to a single value, $v_n$.  An entity also includes additional information such as a unique identifier.

					\begin{figure}[!ht]
						\centering
						\[
							\begin{array}{ll}
								\mathrm{id} & T_n|v_{id} \\
								a_1 & v_1 \\
								a_2 & v_2 \\
								\vdots & \vdots \\
								a_n & v_n
							\end{array}
						\]
						\caption{The structure of an entity}
						\label{fig:entity-rep}
					\end{figure}

					Entities are analogous to rows in a database table.  Thus, the unique identifier is generated based on the table name, $T_n$, as well as unique key in the table, $v_{id}$.  The unique key identifies the row, and the table name identifies the table.  Together they uniquely identify the entity within the entire database.

					$\exists! e_{id} \in E$, where $E$ is the set of all entities in a data set.
				\end{defn}

			\subsection{Entity Group}
				\begin{defn}
					An \textbf{Entity Group}
				\end{defn}

		\section{Ford-Fulkerson}
			\begin{algorithm}
				\begin{algorithmic}
					\ENSURE $1=1$
				\end{algorithmic}
			\end{algorithm}
\end{document}
